%check with Ron before citing his papers because he shared them for internal use only

%%%%% TITLE
\title{Hotspot Analyses with Network and Planar Statistics: When Does It Matter? A Simulated Comparison of Network and Planar Statistics }
\author{Luc Anselin, Sergio Rey, Elizabeth A. Mack, Julia Koschinsky, Ran Wei}
%\affiliation{GeoDa Center for Geospatial Analysis and Computation, School of Geographical Sciences and Urban Planning, Arizona State University}
%\email{luc.anselin@asu.edu}   
\date{\today}


%%%%% PREAMBLE
\documentclass[12pt, letterpaper]{article}
\usepackage[top=1in, bottom=1in, left=1in, right=1in]{geometry}
\usepackage[pdftex]{graphicx}
\usepackage[round]{natbib}
\usepackage{fancyhdr}
\usepackage{setspace}
\usepackage{amsmath} 
\usepackage{titling}


%%%%% DOCUMENT
\begin{document}
\maketitle{}
\setstretch{1.5}
\newpage

\begin{abstract}
Although crime analysts are now able to generate hotspot maps with a variety of techniques, the differences in the hotspots produced by these techniques remains unclear. The quantifiable differences between techniques across a common dataset and across different datasets currently elude analysts. This study attempts to quantify the difference in hotspot maps produced using planar and network methods

\end{abstract}
\newpage

\section{Introduction}

Crime mapping has a long history in criminology. From its humble beginnings  with pin mapping in the 1900's through the computer-generated maps that became commonplace with advancements in computing technology in the 1990's to current real-time mapping applications of crime data; maps are essential crime analysis tools (Harries, 1999). One of the goals of crime mapping is to identify areas of elevated crime or hotspots. Although there is no generally accepted definition of a crime hotspot, a good definition is that "a hotspot is an area that has a greater than average number of criminal or disorder events, or an area where people have a higher than average risk of victimization" (Eck et al, 2005 pg. 2). Currently, there are a plethora of hotspot methods available to crime analysts, which vary by research question and the unit of analysis for which the data are collected. Five general categories of hotspot methods include: choropleth mapping, grid cell analysis, cluster analysis, and spatial autocorrelation (Harries, 1999). The main goal of hotspot mapping is to identify areas with elevated levels of crime so that police and crime reduction efforts may be allocated more efficiently (Ek et al, 2005; Chainey, Tompson, and Uhlig, 2008). 

As crime mapping has increased in sophistication in recent years, so too have the palette of hotspotting methods available to crime analysts. One of the key advancements in hotspotting techniques is the development of methods to evaluate crimes that are constrained to network-space rather than Euclidean space. These advancements are not only important given the inherently different distribution of network-constrained phenomenon, which is distributed within a one-dimensional subset of Euclidean-space \citep{yamada2004comparison, borruso2008network}. They are also important because recent studies have discovered this inherent difference in spatial distribution may cause planar-based cluster statistics to produce false-positives, and detect clusters of events when none are present \citep{yamada2004comparison, luchen2007false, xie2008kernel}. This result is of particular concern from a practical standpoint because it suggests the application of common hotspot tools, like kernel density estimation, may reduce rather than improve policing efficiency. For example, if a police department were using kernel density estimation for more network-constrained crime events, the false positive results may mean they are deploying additional forces where none are required and under-deploying forces where more are needed. The comparison of these methods thus far however has been primarily visual and not empirical. 

Despite this visual demonstration of differences in the results produced by network and planar methods, the circumstances under which these methods are likely to diverge remains unexplored in the literature, and has been recommended as an area for future research \citep{luchen2007false, borruso2008network}. The existing literature suggests differences in these methods are likely to occur when the characteristics of the underlying event distribution and the characteristics of the network vary \citep{yamada2004comparison}. Specifically, it is anticipated the degree of divergence between network and planar methods is likely to be smaller for highly connected street networks and dense points distributions \citep{luchen2007false}because this kind of distribution ``\emph{fills in}'' or approximates Euclidean space. The anticipated difference between the two methods is expected to be greater for less well connected street networks and more sparse point distributions  \citep{luchen2007false}. More importantly, the impact of the characteristics of street networks and point distributions on network-based statistics is not well understood.

In addition to quantifying differences in the results produced by network and planar methods, the circumstances under which these methods are likely to diverge remains under-explored in terms of simulations in the literature, and has been recommended as an area for future research \citep{luchen2007false, borruso2008network}. The purpose of this study is twofold. One, to evaluate the circumstances under which the distribution of events causes the results of planar methods to diverge from network methods. Two, to provide a means of quantifying differences in the hotspots produced by planar and network methods. This latter goal is particularly important given the well known tendency for humans to perceive clusters or differences in mapped data where none exist. Event data will be simulated and the cluster detection abilities of planar methods and network methods evaluated. Specifically, three statistics will be used to evaluate the circumstances that planar methods diverge from network methods: the network kernel density function \citep{okabe2009kernel}, the network local k-function \citep{yamada2007local}, and the network local indicator of spatial association \citep{yamada2010local}. Results are expected to contribute to the theoretical literature on network and planar statistical comparisons, as well as produce practical results about the circumstance under which it is worth taking the additional computation time to use network specific statistics instead of their planar counterparts.


\section{Hotspot Technique Accuracy}%think of a better title here...

Crime forecasting and prediction via hotspots remains an important area of academic inquiry and methodological innovations (Ratcliffe, 2010;  Tompson and Townsley, 2010; Caplan et al; 2010). Although the evolution of techniques for detecting hotspots of crime has advanced tremendously, our understanding of the relative utility of these techniques has not kept pace. Prior studies have visually demonstrated differences in the location, size, and shapes of different techniques, but very few empirical efforts have been made to quantify the differences in the results \citep{chainey2008utility} (Levine, 2008). Therefore, it remains unclear how the application of different techniques to a common dataset, impacts hotspot results, as well as how output varies for a common technique applied to different crime types. Examination of these differences from an empirical perspective, particularly since the goal of hotspots analyses is predictive and prescriptive in nature. 


\subsection{Comparative Hotspot Analyses}
The continued use of hotspot analyses for predictive policing purposes has prompted inquiry into the prediction ability of different hotspotting techniques across different kinds of crimes \citep{chainey2008utility, NIJHillsborough2009, NIJLincoln2009, NIJCharlotte2009}. Although the literature on hotspot analysis (Jefferis, 1999; Chainey et al., 2002; Eck et al., 2005) recognizes the sensitivity of hotspots to the hotspot method utilized few studies have undertaken the task of evaluating the suitability of different hotspotting techniques across different kinds of crime \citep{chainey2008utility}. \citet{chainey2008utility} developed the \emph{prediction accuracy index (PAi)} to evaluate the predictive capacity of hotspotting methods across one type of crime and different kinds of crime. However, their methodology is perhaps invalidated by the use of the boundary of the standard deviational ellipse as the boundary of the hotspot, rather than the minimum bounding polygon or convex hull of the crimes comprising the hotspots (Levine, 2008). In his 2008 response to this study Levine (2008) expands upon their results by evaluating additional hotspotting techniques and by using the area of the convex hull as the area of the hotspot in computing the PAI. Although this index and the methodological approach of this study can be improved upon (Capitan, 2008), the study nonetheless raises important questions about quantifying differences in hotspotting techniques and the varied utility of techniques across crime types.    


\subsection{Network and Planar Comparison}
The need for better mechanisms to quantify the differences in hotspot results is also raised in studies developing network-based methods for analyzing network-constrained data. These studies visually highlight differences in the results generated via planar and network-specific methods (get cites here). Although these studies demonstrate the importance of using network-based methods for the study of particular phenomenon, they also raise several questions. First, although the use of network-based methods is certainly widespread, the kinds of data for which these statistics are most useful remains nebulous. For example, network-events are commonly employed for creating hotspot maps of motor vehicle crimes (get cites here), however, the usefulness of these computationally intensive methods for crimes that are less network-constrained, like burglary and robbery, is uncertain. \citet{yamada2007local} provide some clues about the utility of these statistics via a typology based on the movement and location constraints of various phenomena. However, quantifiable evidence about differences in results between methods for these data remains an area meriting additional research. In addition to the uncertainty about the kinds of data for which network-based methods are most useful, so too are the differences in the results. As mentioned previously, visual output demonstrates that that there appear to be differences in the results, but information about differences in the size, location, and number of hotspots produced by planar and network methods remains largely unquantified. Finally, it is anticipated that the amount of difference between planar and network methods is likely to vary with the street network data and event data used in the analysis. This final question is the focus of this research.

\section{Simulation Literature}
The existing literature suggests differences in these methods are likely to occur when the characteristics of the underlying event distribution and the characteristics of the network vary \citep{yamada2004comparison}. Specifically, it is expected that the degree of divergence between network and planar methods is likely to be smaller for highly connected street networks and dense points distributions \citep{luchen2007false}because dense point distributions ``\emph{fill in}'' or approximate Euclidean space. The anticipated difference between the two methods is expected to be greater for less well connected street networks and more sparse point distributions  \citep{luchen2007false}. Although some empirical research provide some answers to these questions, this is certainly an area worthy of additional research, particularly from a simulation perspective.

Simulation studies

\section{Data}
\subsection{Street Network Data}
\subsection{Simulated Point Data}

\section{Simulation Design}
\subsection{Preprocessing of Street Networks} %remove this since it is from a prior study??
\subsection{Null Hypothesis: Euclidean Space}
\subsection{Null Hypothesis: Network Space}
\subsection{Alternative Hypothesis}

\section{Methods}
%make this abbreviated so it is not a repeat of the first paper....
%Ask about this-should it be shorter than the last piece? Do we want to put equations in here.....

\section{Results}
\section{Discussion and Conclusion}
Quantification of differences in hotspotting techniques is key to proactive policing. This is especially the case when intervention strategies require specific location information for infrastructure improvements, such as the installation of additional lighting, to a crime-ridden street corner (Levine, 2008). As police department budgets are required to do more with fewer funds, so too is the accurate identification of high-crime areas important, rather than an approximate location provided by a kernel-density map.


%%%%%BIBLIOGRAPHY
\newpage
\setstretch{1.0}
\bibliographystyle{apalike}
\bibliography{Event_Simulation2}

\end{document}