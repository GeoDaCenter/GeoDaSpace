\documentclass{article}
\title{GMM Estimation of Spatial Error Autocorrelation with Heteroskedasticity}
\author{Luc Anselin}
\usepackage{amsmath}
\usepackage{natbib}
\bibpunct{(}{)}{,}{a}{}{,}

\begin{document}
\maketitle
\section{Background}
This note documents the steps needed for an efficient GMM estimation of the
regression parameters and autoregressive parameters using the moment conditions
spelled out in \cite{KelejianPrucha:09} and \cite{Arraizetal:10} (referred to in what
follows as K-P). Theoretical details are
provided in those articles. The focus here is on the practical steps to carry out estimation
in a number of special cases. I will be using ``we'' below since all this should eventually
be moved into the spatial econometrics text.

\section{Model Specification and Notation}

The model we consider is the mixed regressive spatial autoregressive model with
a spatial autoregressive error term, or so-called SAR-SAR model. Using the notation from \cite{Anselin:88}, this
is expressed as:
\begin{equation}
    \mathbf{y} = \rho \mathbf{Wy} + \mathbf{X} \beta + \mathbf{u}.
    \label{model1}
\end{equation}

The notation is standard, with $\mathbf{y}$ as a $n \times 1$ vector of observations
on the dependent variable, $\mathbf{W}$ as a $n \times n$ spatial lag operator and
$\mathbf{Wy}$ as the spatial lag term with spatial autoregressive parameter
$\rho$, $\mathbf{X}$ as an $n \times k$ matrix of observations
on explanatory variables with $k \times 1$ coefficient vector $\beta$, and a
$n \times 1$ vector of errors $\mathbf{u}$. The specification is general, in the
sense that $\mathbf{X}$ can contain both exogenous as well as endogenous variables.
In the latter case, a $n \times p$ matrix of instruments $\mathbf{H}$ will be needed.
An alternative way to express the model is as:
\begin{equation}
\mathbf{y} = \mathbf{Z} \delta + \mathbf{u},
\end{equation}
where $\mathbf{Z} = [ \mathbf{X}, \mathbf{Wy} ]$ and the $k \times 1$ coefficient
vector is rearranged as $\delta = [\beta' \rho ]'$ (i.e., a column vector).

The error vector $\mathbf{u}$ follows a spatial autoregressive process:
\begin{equation}
    \mathbf{u} = \lambda \mathbf{W u} + \varepsilon
    \label{model2}
\end{equation}
where $\lambda$ is the spatial autoregressive parameter, and with heteroskedastic 
innovations, such that $\mbox{E}[\varepsilon_i^2] = \sigma_i^2$. All other assumptions
are standard.


Note that, in contrast to the general case outlined by K-P, we take the weights
matrix in the spatial lag and in the spatial error part to be the same ($\mathbf{W}$). In K-P, the
weights matrix for the error term is denoted as $\mathbf{M}$. While this is more
general, it is hardly ever used in practice, hence we limit our treatment to the
simpler case.

\subsection{Spatially Lagged Variables}
Spatially lagged variables play an important part in the GMM estimation procedure.
In the original K-P papers, these are denoted by ``bar'' superscripts. Instead, we will
use the $L$ subscript throughout. In other words, a first order spatial lag of $\mathbf{y}$, 
i.e., $\mathbf{Wy}$ is denoted by $\mathbf{y}_L$, and similarly for spatially lagged
explanatory variables, $\mathbf{X}_L$, and for $\mathbf{Z}_L$. Higher order spatial lags
are symbolized by adding additional $L$ subscripts. For example, a second order 
spatial lag of the error $\mathbf{u}$ would be $\mathbf{u}_{LL}$.


\subsection{Spatial Cochrane-Orcutt Transformation}
An important aspect of the estimation is the use of a set of spatially filtered variables
in a spatially weighted regression. K-P refer to this as a spatial Cochrane-Orcutt transformation.
The spatial filter is based on the weights matrix  and the
spatial autoregressive parameter for the error process. Since there is no
distinction between the two weights here, the matrix $\mathbf{W}$ is used throughout.
In the notation of what follows, we will use a subscript $s$ for the spatially filtered variables. Also, to keep the notation simple, we will not distinguish between the notation
for a parameter and its estimate. In practice, an estimate is always used, since the true
parameter value is unknown.

The spatially filtered variables are then:
\begin{eqnarray}
\mathbf{y}_s &=& \mathbf{y} - \lambda \mathbf{Wy}\\
                    &=& \mathbf{y} - \lambda \mathbf{y}_L\\
 &=& (\mathbf{I} - \lambda \mathbf{W}) \mathbf{y}\\
\mathbf{X}_s &=& \mathbf{X} - \lambda \mathbf{WX}\\
                    &=& \mathbf{X} - \lambda \mathbf{X}_L\\
 &=& (\mathbf{I} - \lambda \mathbf{W}) \mathbf{X}\\
 \mathbf{Wy}_s &=& \mathbf{Wy} - \lambda \mathbf{WWy}\\
                     &=& \mathbf{y_L} - \lambda \mathbf{y}_{LL}\\
 &=& (\mathbf{I} - \lambda \mathbf{W}) \mathbf{Wy}\\
\mathbf{Z}_s &=& \mathbf{Z} - \lambda \mathbf{WZ}\\
                    &=& \mathbf{Z} - \lambda \mathbf{Z}_L\\
 &=& (\mathbf{I} - \lambda \mathbf{W}) \mathbf{Z}
\end{eqnarray}


\section{Outline of the GMM Estimation Procedure}
The GMM estimation is carried out in multiple steps. The basic rationale is the
following. First, an initial estimation yields a set of consistent (but not efficient) estimates
for the model coefficients. For example, in a model with only exogenous explanatory
variables, this would be based on ordinary least squares (OLS). In the presence of
endogenous explanatory variables, two stage least squares (2SLS) would be necessary.
This is also the case when a spatially lagged dependent variable is present. In the latter
case, a number of papers have discussed the use of optimal weights
\cite[e.g.,][]{Lee:03,Dasetal:03,Kelejianetal:04,Lee:07}. In practice, the instruments
consist of the exogenous variables and spatial lags of these, e.g., 
$\mathbf{H} = [ \mathbf{X}, \mathbf{X}_L, \mathbf{X}_{LL}, \dots ]$.

The initial consistent estimates provide the basis for the computation of a vector
of residuals, say $\mathbf{u}$ (here, we do not use separate notation to distinguish the
residuals from the error terms, since we always need residuals in practice). The residuals are 
used in a system of moment equations to provide a first consistent (but not efficient) estimate
for the error autoregressive coefficient $\lambda$. The consistent estimate for $\lambda$ is used
to construct a weighting matrix that is necessary to obtain the optimal (consistent and efficient) GMM
estimate of $\lambda$ in a second iteration.

A third step then consists of estimating the regression coefficients ($\beta$ and $\rho$, if
appropriate) in a spatially weighted regression, using spatially filtered variables that incorporate the
optimal GMM estimate of $\lambda$.

At this point, we could stop the estimation procedure and use the values of the regression
coefficients, the corresponding residuals, and $\lambda$ to construct a joint asymptotic 
variance-covariance matrix for all the coefficients (both regression and $\lambda$).
Alternatively, we could go through another round of estimation using the updated residuals
in the moment equations and potentially going through one more spatially weighted 
regression. While asymptotically, there are no grounds to prefer one over the other, in
practice there may be efficiency gains from further iterations.

Finally, the estimation procedure as outlined in K-P only corrects for the presence of
spatial autoregressive errors, but does not exploit the general structure of the 
heteroskedasticity in the estimation of the regression coefficients. The main contribution
of K-P is to derive the moment equation such that the estimate for $\lambda$ is
consistent in the presence of general heteroskedasticity. The initial GM estimator presented
in \cite{KelejianPrucha:98,KelejianPrucha:99a} is only consistent under the assumption
of absence of heteroskedasticity. We will need to further consider if the incorporation of
both spatial autoregressive and heteroskedastic structures for the error variance 
improves the efficiency of the regression coefficients.

\section{General Moment Equations}
The point of departure for K-P's estimation procedure are two 
moment conditions, expressed as functions of the innovation terms
$\varepsilon$. They are:
\begin{eqnarray}
 n^{-1} \mbox{E} [\varepsilon_L'\varepsilon_L] &=& n^{-1} \mbox{tr} [ \mathbf{W} \mbox{diag}[E(\varepsilon_i^2)] \mathbf{W'} ]\\
 n^{-1} \mbox{E} [\varepsilon_L'\varepsilon] &=& 0,
\end{eqnarray}
where $\varepsilon_L$ is the spatially lagged innovation vector and tr stands for the
matrix trace operator. The main difference
with the moment equations in \cite{KelejianPrucha:99a} is that the innovation vector
is allowed to be heteroskedastic of general form, hence the inclusion of the 
term $\mbox{diag}[E(\varepsilon_i^2)]$. In the absence of heteroskedasticity, this
term simplifies to $\sigma^2 n^{-1} \mbox{tr} [\mathbf{WW'}]$. However, to carry out the
GM estimation in this case, an additional equation is needed for $\sigma^2$.

\subsection{Simplifying Notation}
K-P introduce a number of simplifying notations that allow the moment conditions to
be written in a very concise form. Specifically, they define
\begin{eqnarray}
\mathbf{A}_1 &=& \mathbf{W'W} - \mbox{diag} (\mathbf{w}_{.i}'\mathbf{w}_{.i})\\
\mathbf{A}_2 &=& \mathbf{W},
\end{eqnarray}
where $\mathbf{w_{.i}}$ is the i-th column of the weights matrix $\mathbf{W}$. Upon
further inspection, we see that each element $i$ of the diagonal matrix 
$\mbox{diag} (\mathbf{w}_{.i}'\mathbf{w}_{.i})$ consists of the sum of squares of the
weights in the i-th column. In what follows, we will designate this diagonal matrix
as $\mathbf{D}$.

Using the new notation, the moment conditions become:
\begin{eqnarray}
n^{-1} \mbox{E} [ \varepsilon ' \mathbf{A_1} \varepsilon ] &=& 0\\
n^{-1} \mbox{E} [ \varepsilon ' \mathbf{A_2} \varepsilon ] &=& 0
\end{eqnarray}

In order to operationalize these equations, the (unobservable) innovation terms
$\varepsilon$ are replaced by their counterpart expressed as a function of regression
residuals. Since $\mathbf{u} = \lambda \mathbf{u}_L + \varepsilon$, it follows that
$\varepsilon = \mathbf{u} - \lambda \mathbf{u}_L = \mathbf{u}_s$, the spatially
filtered residuals. The operational form of the moment conditions is
then:
\begin{eqnarray}
n^{-1} \mbox{E} [ \mathbf{u}_s ' \mathbf{A_1} \mathbf{u}_s ] &=& 0\\
n^{-1} \mbox{E} [ \mathbf{u}_s ' \mathbf{A_2} \mathbf{u}_s ] &=& 0
\end{eqnarray}

\subsection{Nonlinear Least Squares}
The initial consistent estimate for $\lambda$ is obtained by solving the moment
conditions for this parameter, which is contained within $\mathbf{u}_s$. The parameter
enters both linearly and in a quadratic form. Since there are two equations,
there is no solution that actually sets the results to zero for both equations.
Consequently, we try to get as close to this as possible and use a least squares
rationale. In other words, if we consider the LHS of the equation as a deviation
from zero, we try to minimize the sum of squared deviations.

In order to implement this in practice, K-P reorganize the equations as 
explicit functions of $\lambda$ and $\lambda^2$. This takes on the general form:
\begin{equation}
\mathbf{m} = \mathbf{g} - \mathbf{G}
\left[
\begin{matrix}
\lambda\\
\lambda^2
\end{matrix}
\right],
\end{equation}
such that an initial estimate of $\lambda$ is obtained as a nonlinear least
squares solution to these equations, $\mbox{argmin}_\lambda (\mathbf{m'm})$.

The vector $\mathbf{g}$ is a $2 \times 1$ vector with the following elements,
as given by K-P:
\begin{eqnarray}
 \mathbf{g}_1 &=& n^{-1} \mathbf{u}' \mathbf{A}_1 \mathbf{u}\\
  \mathbf{g}_2 &=& n^{-1} \mathbf{u}' \mathbf{A}_2 \mathbf{u},
\end{eqnarray}
with the $\mathbf{A}_{1,2}$ as defined above, and $\mathbf{u}$ is a vector
of residuals.

For computational purposes, it is actually more effective to work out these
expressions and turn them into cross products of the residuals and their
spatial lags. This yields:
\begin{eqnarray}
 \mathbf{g}_1 &=& n^{-1} [ \mathbf{u}_L' \mathbf{u}_L - \mathbf{u}' \mathbf{D} \mathbf{u} ]\\
  \mathbf{g}_2 &=& n^{-1} \mathbf{u}' \mathbf{u}_L,
\end{eqnarray}

The matrix $\mathbf{G}$ is a $2 \times 2$ matrix with the following elements,
as given by K-P:
\begin{eqnarray}
\mathbf{G}_{11} &=& 2n^{-1} \mathbf{u}' \mathbf{W'} \mathbf{A}_1 \mathbf{u}\\
\mathbf{G}_{12} &=& - n^{-1} \mathbf{u}' \mathbf{W'} \mathbf{A}_1 \mathbf{W} \mathbf{u}\\
\mathbf{G}_{21} &=& n^{-1} \mathbf{u}' \mathbf{W'} ( \mathbf{A}_2 + \mathbf{A}_2 ' ) \mathbf{u}\\
\mathbf{G}_{22} &=& - n^{-1} \mathbf{u}' \mathbf{W'} \mathbf{A}_2 \mathbf{W} \mathbf{u}
\end{eqnarray}

As before, this simplifies into a number of expressions consisting of cross products
of the residuals and their spatial lags. Specifically, note that $\mathbf{Wu} = \mathbf{u}_L$,
and $\mathbf{WWu} = \mathbf{u}_{LL}$, and, similarly, $\mathbf{u'W'} = \mathbf{u'}_L$
and $\mathbf{u'W'W'} = \mathbf{u'}_{LL}$.

Considering each expression in turn, we find:
\begin{eqnarray}
\mathbf{G}_{11} &=& 2 n^{-1} \mathbf{u'}_L \mathbf{A}_1 \mathbf{u}\\
    &=& 2 n^{-1} [ \mathbf{u'}_{LL} \mathbf{u}_L - \mathbf{u'}_L \mathbf{D} \mathbf{u} ]
\end{eqnarray}

\begin{eqnarray}
\mathbf{G}_{12} &=& - n^{-1} \mathbf{u'}_L \mathbf{A_1} \mathbf{u}_L\\
   &=& - n^{-1} [ \mathbf{u'}_{LL} \mathbf{u}_{LL} - \mathbf{u'}_L \mathbf{D} \mathbf{u}_L ]
\end{eqnarray}

\begin{eqnarray}
 \mathbf{G}_{21} &=& n^{-1} \mathbf{u'}_L (\mathbf{W} + \mathbf{W'}) \mathbf{u}\\
   &=& n^{-1} [ \mathbf{u'}_L \mathbf{u}_L + \mathbf{u'}_{LL} \mathbf{u} ]
\end{eqnarray}

\begin{eqnarray}
 \mathbf{G}_{22} &=& - n^{-1} \mathbf{u'}_L \mathbf{W} \mathbf{u}_L\\
    &=& - n^{-1} \mathbf{u'}_L \mathbf{u}_{LL}
\end{eqnarray}

So, in summary, in order to compute the six elements of $\mathbf{g}$ and $\mathbf{G}$,
we need five cross product terms: $\mathbf{u'} \mathbf{u}_L$, $\mathbf{u'} \mathbf{u}_{LL}$,
$\mathbf{u'}_L \mathbf{u}_L$, $\mathbf{u'}_L \mathbf{u}_{LL}$, and
$\mathbf{u'}_{LL} \mathbf{u}_{LL}$. In addition, we need three weighted cross products:
$\mathbf{u'} \mathbf{D} \mathbf{u}$, $\mathbf{u'}_L \mathbf{D} \mathbf{u}$, and
$\mathbf{u'}_L \mathbf{D} \mathbf{u}_L$ (note that $\mathbf{Du}$ only needs to be
computed once).

Given a vector of residuals (from OLS, 2SLS or even Generalized Spatial 2SLS), the expression
for $\mathbf{g}$ and $\mathbf{G}$ give us a way to obtain a consistent estimate for $\lambda$.

\subsection{Weighted Nonlinear Least Squares}
The estimates for $\lambda$ obtained from the nonlinear least squares are consistent,
but not efficient. Optimal estimates are found from a weighted nonlinear least squares
procedure, or, $\mbox{argmin}_\lambda \mathbf{m'}\Psi^{-1} \mathbf{m}$, where $\Psi$ is a weighting matrix.
The optimal weights correspond to the inverse variance of the moment conditions.

K-P show the general expression for the elements of the $2 \times 2$ matrix $\Psi$ to
be of the form:
\begin{equation}
\psi_{q,r} = (2n)^{-1} \mbox{tr} [ (\mathbf{A}_q + \mathbf{A'}_q ) \Sigma  (\mathbf{A}_r + \mathbf{A'}_r ) \Sigma ] + n^{-1} \mathbf{a'}_q \Sigma \mathbf{a}_r,
\end{equation}
for $q, r = 1,2$ and with $\Sigma$ as a diagonal matrix with as elements 
$(u_i - \lambda u_{L_i})^2 = u_{s_i}^2$, i.e., the squares of the spatially filtered residuals.
The second term in this expression is quite complex, and will be examined more closely below.
However, it is important to note that this term becomes zero when there are only exogenous
explanatory variables in the model (i.e., when OLS is applicable). The term derives from the
expected value of a cross
product of expressions in the $\mathbf{Z}$ matrix and the error term $\mathbf{u}$. Hence,
when no endogenous variables are included in $\mathbf{Z}$, the expected value of this
cross product amounts to $\mbox{E}[ \mathbf{u} ] = 0$.

\subsubsection{OLS Estimation}
In the simplest case when no endogenous variables are present in the model, we
only need to consider the trace term to obtain the elements of $\psi_{q,r}$. Note that 
$\mathbf{A}_1$ is symmetric, so that $\mathbf{A}_1 + \mathbf{A'}_1 = 2 \mathbf{A}_1$.
Also, $\mathbf{A}_2 = \mathbf{W}$ so that we don't need the extra notation at this
point.

Consequently, we obtain the following results:
\begin{eqnarray}
 \psi_{11} &=& (2n)^{-1} \mbox{tr} [ (2 \mathbf{A}_1 ) \Sigma  (2 \mathbf{A}_1 ) \Sigma ]\\
        &=&  2 n^{-1} \mbox{tr} [ \mathbf{A}_1 \Sigma \mathbf{A}_1 \Sigma ],\\
  \psi_{12} &=& (2n)^{-1} \mbox{tr} [ (2 \mathbf{A}_1 ) \Sigma (\mathbf{W} + \mathbf{W'} ) \Sigma ]\\
       &=& n^{-1} \mbox{tr} [  \mathbf{A}_1 \Sigma (\mathbf{W} + \mathbf{W'} ) \Sigma ],\\
   \psi_{21} &=& \psi_{12},\\
   \psi_{22} &=& (2n)^{-1} \mbox{tr} [ (\mathbf{W} + \mathbf{W'}) \Sigma 
   (\mathbf{W} + \mathbf{W'})\Sigma
\end{eqnarray}

For computational purposes, it is important to keep in mind that while the matrices
$\mathbf{A}_{1,2}$ are of dimension $n \times n$, they are typically very sparse.
Furthermore, the matrix $\Sigma$ is a diagonal matrix, such that post-multiplying a matrix by
$\Sigma$ amounts to re-scaling the columns of that matrix by the elements on the diagonal
of $\Sigma$.

\subsubsection{Standard 2SLS Estimation}
The expression for $\mathbf{a}_r$, with $r = 1,2$, for the case where the estimates
are obtained from 2SLS is given in K-P as follows:
\begin{equation}
\mathbf{a}_r = (\mathbf{I} - \lambda \mathbf{W'} )^{-1} \mathbf{HP} \alpha_r,
\end{equation}
with $\mathbf{H}$ as a $n \times p$ matrix of instruments, 
\begin{equation}
\mathbf{P} = (n^{-1}\mathbf{H'H})^{-1} ( n^{-1} \mathbf{H'Z} ) 
        [ (n^{-1} \mathbf{Z'H} ) (n^{-1} \mathbf{H'H} )^{-1} (n^{-1} \mathbf{H'Z}) ]^{-1},
\end{equation}
as a matrix of dimension $p \times k$, and
\begin{equation}
\alpha_r = - n^{-1} [ \mathbf{Z'} (\mathbf{I} - \lambda \mathbf{W'}) (\mathbf{A}_r + \mathbf{A'}_r) 
 ( \mathbf{I} - \lambda \mathbf{W} ) \mathbf{u} ],
\end{equation}
where $\alpha_r$ is a vector of dimension $k \times 1$. As a result, $\mathbf{a}_r$ is of
dimension $n \times 1$.

First, let's take a closer look at $\alpha_r$, for $r = 1,2$. Note that $\mathbf{Z'} (\mathbf{I} - \lambda \mathbf{W'})$ can be written in a simpler form as $\mathbf{Z'}_s$. Similarly, 
$ ( \mathbf{I} - \lambda \mathbf{W} ) \mathbf{u}$ is $\mathbf{u}_s$. For both filtered variables,
we use the value of $\lambda$ from the unweighted nonlinear least squares.

For $\alpha_1$, since $\mathbf{A}_1$ is symmetric, $\mathbf{A}_1 + \mathbf{A'}_1 = 2 \mathbf{A}_1$, and the corresponding expression can be written as:
\begin{eqnarray}
\alpha_1 &=& - 2 n^{-1} [ \mathbf{Z'}_s \mathbf{A}_1 \mathbf{u}_s ]\\
    &=& -2 n^{-1} [ \mathbf{Z'}_{s_L} \mathbf{u}_{s_L} - \mathbf{Z'}_s \mathbf{D} \mathbf{u}_s ],
\end{eqnarray}
where $\mathbf{Z'}_{s_L}$ and $\mathbf{u}_{s_L}$ are the spatial lags (using weights
matrix $\mathbf{W}$) of respectively the spatially filtered $\mathbf{Z}$ and the spatially
filtered $\mathbf{u}$. For $\alpha_2$, the expression is:
\begin{eqnarray}
 \alpha_2 &=& - n^{-1} [ \mathbf{Z'}_s (\mathbf{W} + \mathbf{W'} ) \mathbf{u}_s ] \\
    &=& - n^{-1} [ \mathbf{Z'}_s \mathbf{u}_{s_L} + \mathbf{Z'}_{s_L} \mathbf{u}_s ]
\end{eqnarray}

Pre-multiplying the respective expressions for $\alpha_1$ and $\alpha_2$ with the
matrix $\mathbf{P}$ yields a vector of dimension $p \times 1$ ($p$ is the number of
instruments). At first sight, the presence of the inverse matrix $( \mathbf{I} - \lambda \mathbf{W})^{-1}$ in the expression for $\mathbf{a}_r$ for $r = 1,2$ would seem to preclude large data
analysis. However, we can exploit the approach outlined in \cite{Smirnov:05}. The typical
power expansion (Leontief expansion) of the inverse matrix yields:
\begin{equation}
 ( \mathbf{I} - \lambda \mathbf{W})^{-1} = \mathbf{I} + \lambda \mathbf{W} + \lambda^2 \mathbf{WW} + \dots
\end{equation}
As such, this does not help the computational aspect, since the weights matrices involved are still
of dimension $n \times n$. However, since $( \mathbf{I} - \lambda \mathbf{W})^{-1}$
pre-multiplies $\mathbf{H}$, we see that the multiplication of each term in the expansion with
$\mathbf{H}$ amounts to no more than a series of spatial lag operations, such
as $\lambda \mathbf{WH}$, $\lambda^2 \mathbf{W(WH)}$, etc. Depending on the
value of $\lambda$, a reasonable approximation is readily obtained for a
relatively low power in the expansion.
For example, a value of $\lambda$ of 0.5 (which is relatively large in practice) reduces to
$0.00098$ after a tenth power. Given that this is multiplied with the elements of a
spatial weights matrix, the effective lag operator is even smaller. In our example, assuming
five neighbors on average, this would yield a multiplier in the lag operator of 
$0.00098 / 5 = 0.0002$.\footnote{For high values of $\lambda$, much higher powers
are needed in order to obtain a reasonable approximation. For example, $0.9$ to the tenth power is still 0.35, but
after fifty powers (i.e., fifty lag operations) it is 0.005.}

With these elements in hand, we obtain the terms $\mathbf{a}_1$ and $\mathbf{a}_2$ needed
in the expression $\mathbf{a'}_q \Sigma \mathbf{a}_r$, which, together with the trace
terms, yield the four elements of the matrix $\Psi$.

\subsubsection{Spatially Weighted Estimation}


\bibliography{/users/luc/Papers/lucbibs/papers,/users/luc/Papers/lucbibs/luc}
\bibliographystyle{apalike}


\end{document}
