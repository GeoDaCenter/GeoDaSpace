\section{GeoDa Center} 

\begin{frame}
	\frametitle{GeoDa Center for Geospatial Analysis and Computation}
  \begin{enumerate}
  \item Methods development
  \item Implementation through software tools
  \item Policy relevant research
  \item Dissemination through training and support
  \end{enumerate}
 \end{frame} 

\begin{frame}
	\frametitle{GeoDa Center}
 \begin{itemize}
 \item Arizona State University (beginning fourth year)
 \item Succeeds:
 \begin{itemize}
 \item Spatial Analysis Laboratory (UIUC)
 \item Regional Analysis Laboratory (SDSU)
 \end{itemize}
 \item Five core faculty
 \item Post-docs and graduate students
 \item Multi-disciplinary $\rightarrow$ geographers, computer scientists, economists
 \end{itemize}
 \end{frame} 

\begin{frame}
	\frametitle{Open Source vs. Free Software}
  All GeoDa Center software is free\dots \\ \qquad\qquad\qquad\qquad\qquad\dots some is also open source.
 
\begin{block}{Open Source}
 \begin{itemize}
 \item The raw code is supplied
 \item See how it works
 \item No black box
 \item Can be modified by the user
 \end{itemize}
 \end{block} 
\begin{block}{Free Software}
 \begin{itemize}
 \item No cost to use
 \item Black box
 \item GeoDa Legacy
 \end{itemize}
 \end{block} \end{frame} 

\begin{frame}
	\frametitle{Audience for GeoDa Center Tools}
 \begin{itemize}
 \item Interface
 \begin{itemize}
 \item Point-and-click
 \item Command line
 \end{itemize}
 \item Pedagogical
 \begin{itemize}
 \item Code as text
 \item Non-GIS experts
 \end{itemize}
 \item Audiences
 \begin{itemize}
 \item Social scientists: criminology, demography, urban studies, electoral studies, regional economics, \ldots
 \item Instructors
 \item Applied analysts: police, public health, agriculture, real estate, telco
 \end{itemize}
 \end{itemize}
 \end{frame} 

\begin{frame}
	\frametitle{Today}
 \begin{itemize}
 \item Overview and update of PySAL
 \item Spatial econometrics in PySAL and GeoDaSpace
 \end{itemize}
 \end{frame} 


\section{PySAL} 

\subsection{Background} 

\begin{frame}
	\frametitle{{\color{green}{Py}}thon {\color{green}{S}}patial {\color{green}{A}}nalysis {\color{green}{L}}ibrary}
 
\begin{block}{Leverage Existing Tools Development}
 \begin{itemize}
 \item GeoDa 
 \item STARS
 \item Others
 \end{itemize}
 \end{block} 
\begin{block}{Develop Core Library}
 \begin{itemize}
 \item Spatial data analytical functions 
 \item Enhanced specialization, modularization 
 \item Fill a void in Python libraries 
 \end{itemize}
 \end{block} 
\begin{block}{Flexible Delivery Mechanism}
 \begin{itemize}
 \item GUI, ArcGIS interface 
 \item Spatial analytical web services 
 \end{itemize}
 \end{block} \end{frame} 

\begin{frame}
	\frametitle{Philosophy}
 
\begin{block}{Pure Python}
 \begin{itemize}
 \item Code as text 
 \item Pedagogical goal 
 \end{itemize}
 \end{block} 
\begin{block}{Niche}
 \begin{itemize}
 \item Complementary 
 \item Evolutionary 
 \item Not revolutionary 
 \end{itemize}
 \end{block} \end{frame} 

\begin{frame}
	\frametitle{Functionality -- Big Picture}
  \begin{center}
  \includegraphics<1->[width=0.70\linewidth]{pysalGraphic.png}%
  \end{center}
 \end{frame} 

\subsection{Usage} 

\begin{frame}
	\frametitle{Regular Python Module}
 \VerbatimInput[frame=single,numbers=left,numbersep=3pt,
 fontsize=\small]{figures/pysalStart.txt}
 \end{frame} 

\begin{frame}
	\frametitle{File Input-Output}
 
\begin{block}{File Types}
 \begin{itemize}
 \item .shp $\rightarrow$ read, write
 \item .shx $\rightarrow$ read, write
 \item .dbf $\rightarrow$ read, write
 \item .gwt $\rightarrow$ read
 \item .gal $\rightarrow$ read
 \item .csv $\rightarrow$ read
 \item .wkt $\rightarrow$ read
 \item .geoda\_txt $\rightarrow$ read
 \end{itemize}
 \end{block} \end{frame} 

\begin{frame}
	\frametitle{File Input-Output -- Example (St. Louis Region)}
  \begin{center}
  \includegraphics<1->[width=0.80\linewidth]{stl.png}%
  \end{center}
 \end{frame} 

\begin{frame}
	\frametitle{File Input-Output -- Shapefile (.shp)}
 \VerbatimInput[frame=single,numbers=left,numbersep=3pt,
 fontsize=\small]{figures/pysalFileIO1.txt}
 \end{frame} 

\begin{frame}
	\frametitle{File Input-Output -- Shapefile (.dbf)}
 \VerbatimInput[frame=single,numbers=left,numbersep=3pt,
 fontsize=\small]{figures/pysalFileIO2.txt}
 \end{frame} 

\begin{frame}
	\frametitle{Spatial Weights from Shapefiles}
 \VerbatimInput[frame=single,numbers=left,numbersep=3pt,
 fontsize=\small]{figures/pysalWeights.txt}
 \end{frame} 

\begin{frame}
	\frametitle{Spatial Weights Manipulation}
 \VerbatimInput[frame=single,numbers=left,numbersep=3pt,
 fontsize=\small]{figures/pysalWeights2.txt}
 \end{frame} 


