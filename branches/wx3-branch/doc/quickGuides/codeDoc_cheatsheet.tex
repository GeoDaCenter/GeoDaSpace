% NOTE to compile:
% Since it uses the package 'minted', both 'minted.sty' and 'ifplatform.sty'
% should be in the folder. The command to compile this document should be:

%   pdflatex -shell-escape howto_contrib2cookbook.tex

% Run it TWICE

\documentclass{article}
\usepackage{graphicx}
\usepackage{amsmath}
\usepackage{minted}
\usepackage{hyperref}
\hypersetup{
    colorlinks,
    citecolor=black,
    filecolor=black,
    linkcolor=black,
    urlcolor=blue
}
%\usemintedstyle{autumn}
\title{A quick guide to document code in PySAL and GeoDaSpace}
\begin{document}
\maketitle

%Abstract
\begin{abstract}
    This documents briefly explains the main guidelines followed by the 
    projects \href{http://pysal.org}{PySAL} and GeoDaSpace to handle their
    in-code documentation and testing modules. The main structures and mark-up
    are outlined as well as the principal tricks to know in order to write doctests
    that may be added to the global testing system.
\end{abstract}

\tableofcontents

\newpage

\section{Global guidelines}
This document is a (very) brief and simple description to get started
documenting code in PySAL and GeoDaSpace. By no means does it try to be
comprehensive. It is inspired by the official guidelines followed by the two
projects:

\begin{itemize}
    \item \href{http://projects.scipy.org/numpy/wiki/CodingStyleGuidelines}{Numpy
        documentation style guide}
    \item \href{http://www.python.org/dev/peps/pep-0008/}{General python coding style guide}
\end{itemize}

For a complete description on how to proceed when documenting code, please
refer to them.

\section{Documenting functions and classes}
The purpose of the section is to show how to write documentation within
functions and classes (docstrings); this is the closest sort of help the user will access
and, as such, it needs to be very specific and technical (it is usually other
developers that look at it). There are two main reasons why it is important to
follow a consistent strategy when writing docstrings: first, when projects
grow and more people start contributing, having standards greatly facilitates
readability and organization; second, by using a simple mark-up, the Sphinx
system used by PySAL and GeoDaSpace can process such information and present
it in several output formats (HTML, \LaTeX, etc.) that help dissemination.
Since documenting functions and classes is very similar, we will here show it
at the same time.

\subsection{Location \& demarcation}
\begin{itemize}
    \item Docstrings should be placed \textbf{in the line below the definition
        of the function} ('\texttt{def}') or the class ('\texttt{class}').
    \item The \textbf{first line} should be devoted only to the three quotes
        ('\texttt{"""}') that mark strings in Python.
    \item The \textbf{last line} should also be filled with three quotes to
        mark the end of the string and, preferably, the developer shouls also
        leave a blank line before.
\end{itemize}

\paragraph{Example}

\begin{minted}{python}
def myFunction(x):
    """
    Write the string here

    """
\end{minted}

\subsection{Description}
The docstring should start with a bried description of the functionality of
the method we are writing. It should be followed by a line with three dots
(\texttt{\dots}) and a blank line of separation.

\paragraph{Example}

\begin{minted}{python}
class myClass(x, full=None):
    """
    This is my class which does it all perfectly
    ...

    Rest of the string here

        """
\end{minted}

\subsection{Parameters}
The next section of the string should list all the parameters the
function/class will require to be passed in for it to function. It will be
preceded by the section title (as shown below) and the structure
should be of at least two lines: the first one contains the name of the
parameter plus as many tabs as neccesary to make all of them fit in the same
width, plus the type of object it is (integer, string, etc.); the second line
will contain a description of the parameter, including whether it is optional
or not and, in such case, what the default is.

\paragraph{Example}

\begin{minted}{python}
    """
    ...

    Parameters
    ----------

    x       : array
              Input data
    full    : bolean
              Flag to mark whether or not to use the full case (defaults to
              None)

        """
\end{minted}

\subsection{Attributes}
After especifying what is needed to be passed, you will specify what it comes
out of the method. For classes, more than one attributes is possible, 
functions only return one object. The structure will be the same as for
parameters.

\paragraph{Example}

\begin{minted}{python}
    """
       

    Attributes
    ----------

    res     : list
              List with the output result, the number of trials it took and
              the time, in that order
    x       : array
              Original input data

        """
\end{minted}

\subsection{References}
If you cite any publication in the docstring, you can add a section on
references that lists the full citation at the end of it. Then you can refer
to it within the docstring, similarly as you would do, for instance, with
BibTex and \LaTeX. When Sphinx processes the documentation, it will created
hypelinks in the text that will link citation and reference.

\paragraph{Example}

\begin{itemize}
    \item Within the docstring you will cite the publication in the format you
        want (as you would do in a paper) and add its \textit{reference
        number}. If this was your first reference, you would write it like
        this:
\begin{minted}{python}
def myFunction(x):
    """
    Lagrange Multiplier tests. Implemented as presented in Anselin et al.
    (1996) [1]_
    ...

    """
\end{minted}
\item Then at the end, just before the 'Examples' section you would add
    another section:
\begin{minted}{python}
    """
    ...

    References
    ----------

    .. [1] Anselin, L., Bera, A. K., Florax, R., Yoon, M. J. (1996) "Simple
       diagnostic tests for spatial dependence". Regional Science and Urban
       Economics, 26, 77-104.

        """
\end{minted}

\end{itemize}

\subsection{Examples}
This section has a double purpose: on a pedagogical side, it is useful to show
the user how the function would be used as well as what she should expect from
it; on the other hand, it plays a crucial role for the tests. This is the part
the testing module looks at to run the tests and make sure new changes do not
break previous code.

The structure replicates the python interpreter and shows both the input and
the output expected in case the function/class was used. Both will have to be
hardcoded and it is very important that, before adding it, you
\textbf{make sure the result of the example is correct}. The input should be
preceded by the '\texttt{>>>}' and the output with nothing. The example must be
completely replicable (in fact, that is exactly what the testing module does),
that means in must include reading any kind of data you have to use.
It should be written as shown in the example. When writing an example, try to
put in as many cases as possible (in the case of classes, as many attributes
as possible) so the test covers as much functionality as possible.

\begin{minted}{python}
    """

    Examples
    --------

    >>> import numpy as np
    >>> x = pysal.open('Xfile.csv', 'r')
    >>> x = np.array(x)
    >>> c = myClass(x, full=True)
    >>> c.res
    [3.44426, 7, 0.0031]
    >>> type(c.x)
    <type 'numpy.ndarray'>

        """
\end{minted}

\newpage

\subsection{Template example}
Here is what a full well documented docstring should look like:

\begin{minted}{python}
class myClass(x, full=None):
    """
    This is my class which does it all perfectly, as in Jones and Jacobs
    (1997) [1]_
    ...

    Parameters
    ----------

    x       : array
              Input data
    full    : bolean
              Flag to mark whether or not to use the full case (defaults to
              None)

    Attributes
    ----------

    res     : list
              List with the output result, the number of trials it took and
              the time, in that order
    x       : array
              Original input data

    References
    ----------

    .. [1] Jones, G. and Jacobs, W. (1997) "The comic book heroes". Prima
    Publ.
 
    Examples
    --------

    >>> import numpy as np
    >>> x = pysal.open('Xfile.csv', 'r')
    >>> x = np.array(x)
    >>> c = myClass(x, full=True)
    >>> c.res
    [3.44426, 7, 0.0031]
    >>> type(c.x)
    <type 'numpy.ndarray'>

        """
\end{minted}

\newpage

\section{Testing module}
When you are done documenting the code you have written and are ready to push
it to the PySAL or GeoDaSpace repository, there is one more step you need to
take before hitting the \texttt{svn commit}: make you what you are about to
add does not break any other component of the library. Here is where the
testing module comes in handy. There are two types of tests you may run, local
and global:

\subsection{Local test} 
This will run all the tests in the module where you have written the code. All
you need to do is run the module in the command line (here we assume you have
added lines in the module \texttt{ols.py}:

\begin{minted}{bash}
> python ols.py
\end{minted}

If there is something wrong, this will raise an error that will guide you
about where it breaks; if everything goes fine, no message will come up.

\subsection{Global tests}
The real power of the testing system appears when it comes to checking your
changes haven't broken anything in other modules of the library. The global
tests run all the examples in the library to make sure everything is still
working fine. To run them, you will have to go to the folder where the module
\texttt{tests.py} is located and run it:

\begin{minted}{bash}
> python tests.py
\end{minted}

\subsection{Setting up local tests in a new module}
In case you have created a brand new module, adding the testing system to it
is very simple. Once all the classes and functions are well documented with
correct examples included, all you have to do at the end of the module is
adding the following:

\begin{itemize}
    \item In the case of PySAL:
    \begin{minted}{python}
    if __name__ == "__main__":

        import doctest
        doctest.testmod()
    \end{minted}
\item For GeoDaSpace:
    \begin{minted}{python}
    if __name__ == '__main__':
        _test()
    \end{minted}
\end{itemize}

\subsection{Setting up a new global testing module}
Placeholder.

\appendix
\section{Authors}
Put you name here if you have contributed to this document.
\begin{itemize}
    \item Daniel Arribas-Bel
       \href{mailto:darribas@asu.edu}{\texttt{<darribas@asu.edu>}}
\end{itemize}
\end{document}

