\section{Spatial Econometrics} 

\subsection{Background} 

\begin{frame}
	\frametitle{Spatial Econometric Tools}
 \begin{itemize}
 \item Not much commercially available
 \item Many specialized scripts
 \begin{itemize}
 \item Stata, SAS, SPSS, etc.
 \end{itemize}
 \item Toolboxes and libraries
 \begin{itemize}
 \item R: spdep, sphet
 \item MatLab: Spatial Econometrics Toolbox
 \end{itemize}
 \end{itemize}
 \end{frame} 

\begin{frame}
	\frametitle{Econometrics in Python -- Options}
 \begin{itemize}
 \item pyGauss
 \item pyTrix
 \item EconPy
 \item statsmodels / pandas
 \end{itemize}
 \end{frame} 

\begin{frame}
	\frametitle{Econometrics in Python -- Future}
  \begin{quote}
  ``Since there is no free programming language that can be
  considered a \emph{lingua franca} of applied econometrics, choosing
  Python and writing or translating econometric routines may be
  worth the effort.''
  \end{quote}
  \qquad\qquad\qquad\qquad -- C. Choirat and R. Seri (2009), J. Appl. Econ.
 \end{frame} 

\begin{frame}
	\frametitle{Immediate Plan}
 \begin{itemize}
     \item OLS estimation with \textcolor{blue}{diagnostics for spatial effects}
     \item 2SLS estimation with \textcolor{blue}{diagnostics for spatial effects}
     \item Spatial 2SLS for \textcolor{blue}{spatial lag model} (with endogeneity)
 \item GM and GMM estimation for \textcolor{blue}{spatial error model}
 \item GMM spatial error with \textcolor{blue}{heteroskedasticity}
 \item \textcolor{blue}{Spatial HAC} estimation
 \end{itemize}
 \end{frame} 

\begin{frame}
	\frametitle{Challenges}
 \begin{itemize}
 \item Handle large problems
 \item Efficient spatial weights
 \item Modularity and reusability
 \end{itemize}
 \end{frame} 

\begin{frame}
	\frametitle{Functionality}
 \begin{itemize}
 \item Spatial weights creation
 \item Spatially lagged variable computation
 \item GMM estimation methods
 \item Diagnostics
 \item Allow endogeneity
 \end{itemize}
 \end{frame} 

\begin{frame}
	\frametitle{Delivery}
 \begin{itemize}
 \item Freestanding $\rightarrow$ GeoDaSpace
 \item Command line $\rightarrow$ PySAL
 \item ArcGIS toolbox
 \end{itemize}
 \end{frame} 

\subsection{PySAL} 

\begin{frame}
	\frametitle{PySAL Econometrics -- Setup OLS}
 \VerbatimInput[frame=single,numbers=left,numbersep=3pt,
 fontsize=\small]{figures/ols_commandLine.txt}
 \end{frame} 

\begin{frame}
	\frametitle{PySAL Econometrics -- Diagnostics Summary}
  \begin{center}
  \includegraphics<1->[width=0.53\linewidth]{ols_summary.png}%
  \end{center}
 \end{frame} 

\begin{frame}
	\frametitle{PySAL Econometrics -- Individual Diagnostics}
 \VerbatimInput[frame=single,numbers=left,numbersep=3pt,
 fontsize=\small]{figures/ols_diag.txt}
 \end{frame} 

\begin{frame}
	\frametitle{PySAL Econometrics -- Spatial Diagnostics}
 \VerbatimInput[frame=single,numbers=left,numbersep=3pt,
 fontsize=\small]{figures/spatial_diag.txt}
 \end{frame} 

\begin{frame}
	\frametitle{PySAL Econometrics -- Setup S2SLS}
 \VerbatimInput[frame=single,numbers=left,numbersep=3pt,
 fontsize=\small]{figures/s2sls_setup.txt}
 \end{frame} 

\begin{frame}
	\frametitle{PySAL Econometrics -- Setup GSLS (KP 1998, 1999)}
 \VerbatimInput[frame=single,numbers=left,numbersep=3pt,
 fontsize=\small]{figures/gsls_setup.txt}
 \end{frame} 

\subsection{GeoDaSpace} 

\begin{frame}
	\frametitle{Loosely Coupled Framework}
  \begin{center}
  \includegraphics<1->[width=0.70\linewidth]{software_links.png}%
  \end{center}
 \end{frame} 

\begin{frame}
	\frametitle{User Interface}
  \begin{center}
  \includegraphics<1->[width=0.60\linewidth]{space1.png}%
  \llap{\includegraphics<2->[width=0.60\linewidth]{space2.png}}%
  \llap{\includegraphics<3->[width=0.50\linewidth]{spaceW.png}}%
  \llap{\includegraphics<4->[width=0.60\linewidth]{space3.png}}%
  \llap{\includegraphics<5->[width=0.20\linewidth]{spaceL.png}}%
  \llap{\includegraphics<6->[width=0.60\linewidth]{space4.png}}%
  \llap{\includegraphics<6->[width=0.20\linewidth]{spaceL.png}}%
  \llap{\includegraphics<7->[width=0.60\linewidth]{space5.png}}%
  \llap{\includegraphics<8->[width=0.80\linewidth]{spaceR.png}}%
  \llap{\includegraphics<9->[width=0.60\linewidth]{space4.png}}%
  \end{center}
 \end{frame} 

\begin{frame}
	\frametitle{(Near) Future}
 \begin{itemize}
 \item Spatial regimes
 \item Probit (classic and spatial)
 \item Maximum Likelihood
 \item Panel data
 \end{itemize}
 \end{frame} 



\section{Conclusion} 

\subsection{Next Steps} 

\begin{frame}
	\frametitle{PySAL Release Schedule}
	\begin{block}{6 month release cycle}
 \begin{itemize}
 \item 1.0 - August 1, 2010
        \begin{itemize}
	  \item (700+ downloads since)
	\end{itemize}
 \item 1.1 - January 31, 2011
        \begin{itemize}
	  \item Spatial Dynamics
	  \item Space-Time Event Clustering
	  \item \alert{Spatial Econometrics}
	\end{itemize}
\item 2.0 - August 1, 2011
 \end{itemize}
 \end{block}
 \end{frame} 

\begin{frame}
	\frametitle{Other GeoDa Center Software}
 
\begin{block}{Currently Available}
 \begin{itemize}
 \item PySAL
 \item GeoDa (legacy)
 \item OpenGeoDa
 \item STARS
 \end{itemize}
 \end{block} 
\begin{block}{Coming Soon}
 \begin{itemize}
 \item GeoDaSpace
 \item GeoDaNet
 \item GeoDaWeight
 \item dynTM
 \end{itemize}
 \end{block} \end{frame} 

\begin{frame}
	\frametitle{Questions}
  \begin{center}
 \begin{figure}[htbp]
 \includegraphics[width=0.60\linewidth]{pysalGraphic.png}
 \end{figure}
  {\color{blue}{\Large{geodacenter.asu.edu}}}\\
  {\color{blue}{\Large{www.pysal.org}}}
 \end{center}
 \end{frame} 



