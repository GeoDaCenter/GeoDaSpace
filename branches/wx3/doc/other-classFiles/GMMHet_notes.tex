\documentclass{article}
\title{GMM Estimation of Spatial Error Autocorrelation with Heteroskedasticity}
\author{Luc Anselin}
\usepackage{amsmath}
\usepackage{natbib}
\bibpunct{(}{)}{,}{a}{}{,}

\begin{document}
\maketitle
\section{Background}
This note documents the steps needed for an efficient GMM estimation of the
regression parameters and autoregressive parameters using the moment conditions
spelled out in \cite{KelejianPrucha:09} and \cite{Arraizetal:10} (jointly referred to in what
follows as K-P). Theoretical details are
provided in those articles. The focus here is on the practical steps to carry out estimation
in a number of special cases. I will be using ``we'' below since all this should eventually
be moved into the Anselin-Rey spatial econometrics text.

\section{Model Specification and Notation}

The model we consider is the mixed regressive spatial autoregressive model with
a spatial autoregressive error term, or so-called SAR-SAR model. Using the notation from \cite{Anselin:88}, this
is expressed as:
\begin{equation*}
    \mathbf{y} = \rho \mathbf{Wy} + \mathbf{X} \beta + \mathbf{u}.
    \label{model1}
\end{equation*}

The notation is standard, with $\mathbf{y}$ as a $n \times 1$ vector of observations
on the dependent variable, $\mathbf{W}$ as a $n \times n$ spatial lag operator and
$\mathbf{Wy}$ as the spatial lag term with spatial autoregressive parameter
$\rho$, $\mathbf{X}$ as an $n \times k$ matrix of observations
on explanatory variables with $k \times 1$ coefficient vector $\beta$, and a
$n \times 1$ vector of errors $\mathbf{u}$. The specification is general, in the
sense that $\mathbf{X}$ can contain both exogenous as well as endogenous variables.
In the latter case, a $n \times p$ matrix of instruments $\mathbf{H}$ will be needed.
An alternative way to express the model is as:
\begin{equation}\label{eq:generic}
\mathbf{y} = \mathbf{Z} \delta + \mathbf{u},
\end{equation}
where $\mathbf{Z} = [ \mathbf{X}, \mathbf{Wy} ]$ and the $k \times 1$ coefficient
vector is rearranged as $\delta = [\beta' \rho ]'$ (i.e., a column vector).

The error vector $\mathbf{u}$ follows a spatial autoregressive process:
\begin{equation*}
    \mathbf{u} = \lambda \mathbf{W u} + \varepsilon
    \label{model2}
\end{equation*}
where $\lambda$ is the spatial autoregressive parameter, and with heteroskedastic 
innovations, such that $\mbox{E}[\varepsilon_i^2] = \sigma_i^2$. All other assumptions
are standard.


Note that, in contrast to the general case outlined by K-P, we take the weights
matrix in the spatial lag and in the spatial error part to be the same ($\mathbf{W}$). In K-P, the
weights matrix for the error term is denoted as $\mathbf{M}$. While this is more
general, it is hardly ever used in practice, hence we limit our treatment to the
simpler case.

\subsection{Spatially Lagged Variables}
Spatially lagged variables play an important part in the GMM estimation procedure.
In the original K-P papers, these are denoted by ``bar'' superscripts. Instead, we will
use the $L$ subscript throughout. In other words, a first order spatial lag of $\mathbf{y}$, 
i.e., $\mathbf{Wy}$ is denoted by $\mathbf{y}_L$, and similarly for spatially lagged
explanatory variables, $\mathbf{X}_L$, and for $\mathbf{Z}_L$. Higher order spatial lags
are symbolized by adding additional $L$ subscripts. For example, a second order 
spatial lag of the error $\mathbf{u}$ would be $\mathbf{u}_{LL}$.


\subsection{Spatial Cochrane-Orcutt Transformation}
An important aspect of the estimation is the use of a set of spatially filtered variables
in a spatially weighted regression. K-P refer to this as a spatial Cochrane-Orcutt transformation.
The spatial filter is based on the weights matrix  and the
spatial autoregressive parameter for the error process. Since there is no
distinction between the two weights here, the matrix $\mathbf{W}$ is used throughout.
In the notation of what follows, we will use a subscript $s$ for the spatially filtered variables. Also, to keep the notation simple, we will not distinguish between the notation
for a parameter and its estimate. In practice, an estimate is always used, since the true
parameter value is unknown.

The spatially filtered variables are then:
\begin{eqnarray*}
\mathbf{y}_s &=& \mathbf{y} - \lambda \mathbf{Wy}\\
                    &=& \mathbf{y} - \lambda \mathbf{y}_L\\
 &=& (\mathbf{I} - \lambda \mathbf{W}) \mathbf{y}\\
\mathbf{X}_s &=& \mathbf{X} - \lambda \mathbf{WX}\\
                    &=& \mathbf{X} - \lambda \mathbf{X}_L\\
 &=& (\mathbf{I} - \lambda \mathbf{W}) \mathbf{X}\\
 \mathbf{Wy}_s &=& \mathbf{Wy} - \lambda \mathbf{WWy}\\
                     &=& \mathbf{y}_L - \lambda \mathbf{y}_{LL}\\
 &=& (\mathbf{I} - \lambda \mathbf{W}) \mathbf{Wy}\\
\mathbf{Z}_s &=& \mathbf{Z} - \lambda \mathbf{WZ}\\
                    &=& \mathbf{Z} - \lambda \mathbf{Z}_L\\
 &=& (\mathbf{I} - \lambda \mathbf{W}) \mathbf{Z}
\end{eqnarray*}


\section{Outline of the GMM Estimation Procedure}
The GMM estimation is carried out in multiple steps. The basic rationale is the
following. First, an initial estimation yields a set of consistent (but not efficient) estimates
for the model coefficients. For example, in a model with only exogenous explanatory
variables, this would be based on ordinary least squares (OLS). In the presence of
endogenous explanatory variables, two stage least squares (2SLS) would be necessary.
This is also the case when a spatially lagged dependent variable is present. In the latter
case, a number of papers have discussed the use of optimal instruments
\cite[e.g.,][]{Lee:03,Dasetal:03,Kelejianetal:04,Lee:07}. In practice, the instruments
consist of the exogenous variables and spatial lags of these, e.g., 
$\mathbf{H} = [ \mathbf{X}, \mathbf{X}_L, \mathbf{X}_{LL}, \dots ]$.

The initial consistent estimates provide the basis for the computation of a vector
of residuals, say $\mathbf{u}$ (here, we do not use separate notation to distinguish the
residuals from the error terms, since we always need residuals in practice). The residuals are 
used in a system of moment equations to provide a first consistent (but not efficient) estimate
for the error autoregressive coefficient $\lambda$. The consistent estimate for $\lambda$ is used
to construct a weighting matrix that is necessary to obtain the optimal (consistent and efficient) GMM
estimate of $\lambda$ in a second iteration.

A third step then consists of estimating the regression coefficients ($\beta$ and $\rho$, if
appropriate) in a spatially weighted regression, using spatially filtered variables that incorporate the
optimal GMM estimate of $\lambda$.

At this point, we could stop the estimation procedure and use the values of the regression
coefficients, the corresponding residuals, and $\lambda$ to construct a joint asymptotic 
variance-covariance matrix for all the coefficients (both regression and $\lambda$).
Alternatively, we could go through another round of estimation using the updated residuals
in the moment equations and potentially going through one more spatially weighted 
regression. While, asymptotically, there are no grounds to prefer one over the other, in
practice there may be efficiency gains from further iterations.

Finally, the estimation procedure as outlined in K-P only corrects for the presence of
spatial autoregressive errors, but does not exploit the general structure of the 
heteroskedasticity in the estimation of the regression coefficients. The main contribution
of K-P is to derive the moment equation such that the estimate for $\lambda$ is
consistent in the presence of general heteroskedasticity. The initial GM estimator presented
in \cite{KelejianPrucha:98,KelejianPrucha:99a} is only consistent under the assumption
of absence of heteroskedasticity. We will need to further consider if the incorporation of
both spatial autoregressive and heteroskedastic structures for the error variance 
in a feasible generalized least squares procedure (FGLS)
improves the efficiency of the regression coefficients.

\section{General Moment Equations}
The point of departure for K-P's estimation procedure are two 
moment conditions, expressed as functions of the innovation terms
$\varepsilon$. They are:
\begin{eqnarray}
 n^{-1} \mbox{E} [\varepsilon_L'\varepsilon_L] &=& n^{-1} \mbox{tr} [ \mathbf{W} \mbox{diag}[E(\varepsilon_i^2)] \mathbf{W'} ] \label{eq:firstmoment}\\ 
 n^{-1} \mbox{E} [\varepsilon_L'\varepsilon] &=& 0,
\end{eqnarray}
where $\varepsilon_L$ is the spatially lagged innovation vector and tr stands for the
matrix trace operator. The main difference
with the moment equations in \cite{KelejianPrucha:99a} is that the innovation vector
is allowed to be heteroskedastic of general form, hence the inclusion of the 
term $\mbox{diag}[E(\varepsilon_i^2)]$ in Equation~\ref{eq:firstmoment}. In the absence of heteroskedasticity, the
RHS of the first condition simplifies to $\sigma^2 n^{-1} \mbox{tr} [\mathbf{WW'}]$. However, to carry out the
GM estimation in this case, an additional equation is needed for $\sigma^2$.

\subsection{Simplifying Notation}
K-P introduce a number of simplifying notations that allow the moment conditions to
be written in a very concise form. Specifically, they define
\begin{eqnarray*}
\mathbf{A}_1 &=& \mathbf{W'W} - \mbox{diag} (\mathbf{w}_{.i}'\mathbf{w}_{.i})\\
\mathbf{A}_2 &=& \mathbf{W},
\end{eqnarray*}
where $\mathbf{w_{.i}}$ is the i-th column of the weights matrix $\mathbf{W}$. Upon
further inspection, we see that each element $i$ of the diagonal matrix 
$\mbox{diag} (\mathbf{w}_{.i}'\mathbf{w}_{.i})$ consists of the sum of squares of the
weights in the i-th column. In what follows, we will designate this diagonal matrix
as $\mathbf{D}$.

Using the new notation, the moment conditions become:
\begin{eqnarray*}
n^{-1} \mbox{E} [ \varepsilon ' \mathbf{A_1} \varepsilon ] &=& 0\\
n^{-1} \mbox{E} [ \varepsilon ' \mathbf{A_2} \varepsilon ] &=& 0
\end{eqnarray*}

In order to operationalize these equations, the (unobservable) innovation terms
$\varepsilon$ are replaced by their counterpart expressed as a function of regression
residuals. Since $\mathbf{u} = \lambda \mathbf{u}_L + \varepsilon$, it follows that
$\varepsilon = \mathbf{u} - \lambda \mathbf{u}_L = \mathbf{u}_s$, the spatially
filtered residuals. The operational form of the moment conditions is
then:
\begin{eqnarray}
n^{-1} \mbox{E} [ \mathbf{u}_s ' \mathbf{A_1} \mathbf{u}_s ] &=& 0 \label{eq:firstmoment1}\\
n^{-1} \mbox{E} [ \mathbf{u}_s ' \mathbf{A_2} \mathbf{u}_s ] &=& 0 \label{eq:secondmoment1}
\end{eqnarray}

\subsection{Nonlinear Least Squares}\label{ss:nonlinearls}
The initial consistent estimate for $\lambda$ is obtained by solving the moment
conditions in Equations~\ref{eq:firstmoment1} and \ref{eq:secondmoment1} for this parameter, which is contained within $\mathbf{u}_s$. The parameter
enters both linearly and in a quadratic form. Since there are two equations,
there is no solution that actually sets the results to zero for both equations.
Consequently, we try to get as close to this as possible and use a least squares
rationale. In other words, if we consider the LHS of the equation as a deviation
from zero, we try to minimize the sum of squared deviations.

In order to implement this in practice, K-P reorganize the equations as 
explicit functions of $\lambda$ and $\lambda^2$. This takes on the general form:
\begin{equation*}
\mathbf{m} = \mathbf{g} - \mathbf{G}
\left[
\begin{matrix}
\lambda\\
\lambda^2
\end{matrix}
\right] = 0,
\end{equation*}
such that an initial estimate of $\lambda$ is obtained as a nonlinear least
squares solution to these equations, $\mbox{argmin}_\lambda (\mathbf{m'm})$.

The vector $\mathbf{g}$ is a $2 \times 1$ vector with the following elements,
as given by K-P:
\begin{eqnarray*}
 \mathbf{g}_1 &=& n^{-1} \mathbf{u}' \mathbf{A}_1 \mathbf{u}\\
  \mathbf{g}_2 &=& n^{-1} \mathbf{u}' \mathbf{A}_2 \mathbf{u},
\end{eqnarray*}
with the $\mathbf{A}_{1,2}$ as defined above, and $\mathbf{u}$ is a vector
of residuals. Note that there is actually no gain in using the notation $\mathbf{A}_2$,
since it is the same as $\mathbf{W}$.

For computational purposes, it is useful to work out these
expressions and express them as cross products of the residuals and their
spatial lags. This yields:
\begin{eqnarray*}
 \mathbf{g}_1 &=& n^{-1} [ \mathbf{u}_L' \mathbf{u}_L - \mathbf{u}' \mathbf{D} \mathbf{u} ]\\
  \mathbf{g}_2 &=& n^{-1} \mathbf{u}' \mathbf{u}_L,
\end{eqnarray*}

The matrix $\mathbf{G}$ is a $2 \times 2$ matrix with the following elements,
as given by K-P:
\begin{eqnarray*}
\mathbf{G}_{11} &=& 2n^{-1} \mathbf{u}' \mathbf{W'} \mathbf{A}_1 \mathbf{u}\\
\mathbf{G}_{12} &=& - n^{-1} \mathbf{u}' \mathbf{W'} \mathbf{A}_1 \mathbf{W} \mathbf{u}\\
\mathbf{G}_{21} &=& n^{-1} \mathbf{u}' \mathbf{W'} ( \mathbf{A}_2 + \mathbf{A}_2 ' ) \mathbf{u}\\
\mathbf{G}_{22} &=& - n^{-1} \mathbf{u}' \mathbf{W'} \mathbf{A}_2 \mathbf{W} \mathbf{u}
\end{eqnarray*}

As before, this simplifies into a number of expressions consisting of cross products
of the residuals and their spatial lags. Specifically, note that $\mathbf{Wu} = \mathbf{u}_L$,
and $\mathbf{WWu} = \mathbf{u}_{LL}$, and, similarly, $\mathbf{u'W'} = \mathbf{u'}_L$
and $\mathbf{u'W'W'} = \mathbf{u'}_{LL}$.

Considering each expression in turn, we find:
\begin{eqnarray}
\mathbf{G}_{11} &=& 2 n^{-1} \mathbf{u'}_L \mathbf{A}_1 \mathbf{u}\label{eq:G11}\\
    &=& 2 n^{-1} [ \mathbf{u'}_{LL} \mathbf{u}_L - \mathbf{u'}_L \mathbf{D} \mathbf{u} ]\nonumber
\end{eqnarray}

\begin{eqnarray}
\mathbf{G}_{12} &=& - n^{-1} \mathbf{u'}_L \mathbf{A_1} \mathbf{u}_L\label{eq:G12}\\
   &=& - n^{-1} [ \mathbf{u'}_{LL} \mathbf{u}_{LL} - \mathbf{u'}_L \mathbf{D} \mathbf{u}_L ]\nonumber
\end{eqnarray}

\begin{eqnarray}
 \mathbf{G}_{21} &=& n^{-1} \mathbf{u'}_L (\mathbf{W} + \mathbf{W'}) \mathbf{u} \label{eq:G21}\\
   &=& n^{-1} [ \mathbf{u'}_L \mathbf{u}_L + \mathbf{u'}_{LL} \mathbf{u} ]\nonumber
\end{eqnarray}

\begin{eqnarray}
 \mathbf{G}_{22} &=& - n^{-1} \mathbf{u'}_L \mathbf{W} \mathbf{u}_L \label{eq:G22}\\
    &=& - n^{-1} \mathbf{u'}_L \mathbf{u}_{LL}\nonumber
\end{eqnarray}

So, in summary, in order to compute the six elements of $\mathbf{g}$ and $\mathbf{G}$,
we need five cross product terms: $\mathbf{u'} \mathbf{u}_L$, $\mathbf{u'} \mathbf{u}_{LL}$,
$\mathbf{u'}_L \mathbf{u}_L$, $\mathbf{u'}_L \mathbf{u}_{LL}$, and
$\mathbf{u'}_{LL} \mathbf{u}_{LL}$. In addition, we need three weighted cross products:
$\mathbf{u'} \mathbf{D} \mathbf{u}$, $\mathbf{u'}_L \mathbf{D} \mathbf{u}$, and
$\mathbf{u'}_L \mathbf{D} \mathbf{u}_L$ (note that $\mathbf{Du}$ only needs to be
computed once). Alternatively, if the matrix $\mathbf{A}_1$ is stored efficiently in sparse form,
we can use the cross products $\mathbf{u'}\mathbf{A}_1 \mathbf{u}$, $\mathbf{u'}_L \mathbf{A}_1 \mathbf{u}$ and $\mathbf{u'}_L \mathbf{A}_1 \mathbf{u}_L$.

Given a vector of residuals (from OLS, 2SLS or even Generalized Spatial 2SLS), the expression
for $\mathbf{g}$ and $\mathbf{G}$ give us a way to obtain a consistent estimate for $\lambda$.

\subsection{Weighted Nonlinear Least Squares}
The estimates for $\lambda$ obtained from the nonlinear least squares are consistent,
but not efficient. Optimal estimates are found from a weighted nonlinear least squares
procedure, or, $\mbox{argmin}_\lambda \mathbf{m'}\mathbf{\mathbf{\Psi}}^{-1} \mathbf{m}$, where $\mathbf{\Psi}$ is a weighting matrix.
The optimal weights correspond to the inverse variance of the moment conditions.

K-P show the general expression for the elements of the $2 \times 2$ matrix $\mathbf{\Psi}$ to
be of the form:
\begin{equation}\label{eq:psiqr}
\psi_{q,r} = (2n)^{-1} \mbox{tr} [ (\mathbf{A}_q + \mathbf{A'}_q ) \mathbf{\mathbf{\Sigma}}  (\mathbf{A}_r + \mathbf{A'}_r ) \mathbf{\Sigma} ] + n^{-1} \mathbf{a'}_q \mathbf{\Sigma} \mathbf{a}_r,
\end{equation}
for $q, r = 1,2$ and with $\mathbf{\Sigma}$ as a diagonal matrix with as elements 
$(u_i - \lambda u_{L_i})^2 = u_{s_i}^2$, i.e., the squares of the spatially filtered residuals.
The second term in this expression is quite complex, and will be examined more closely 
in Sections~\ref{ss:weightsstandard} and \ref{ss:weightsfilter} below.
However, it is important to note that this term becomes zero when there are only exogenous
explanatory variables in the model (i.e., when OLS is applicable). The term derives from the
expected value of a cross
product of expressions in the $\mathbf{Z}$ matrix and the error term $\mathbf{u}$. Hence,
when no endogenous variables are included in $\mathbf{Z}$, the expected value of this
cross product amounts to $\mbox{E}[ \mathbf{u} ] = 0$.

\subsubsection{OLS Estimation}
In the simplest case when no endogenous variables are present in the model, we
only need to consider the trace term to obtain the elements of $\psi_{q,r}$. Note that 
$\mathbf{A}_1$ is symmetric, so that $\mathbf{A}_1 + \mathbf{A'}_1 = 2 \mathbf{A}_1$.
Also, $\mathbf{A}_2 = \mathbf{W}$ so that we don't need the extra notation at this
point.

Consequently, we obtain the following results:
\begin{eqnarray*}
 \psi_{11} &=& (2n)^{-1} \mbox{tr} [ (2 \mathbf{A}_1 ) \mathbf{\Sigma}  (2 \mathbf{A}_1 ) \mathbf{\Sigma} ]\\
        &=&  2 n^{-1} \mbox{tr} [ \mathbf{A}_1 \mathbf{\Sigma} \mathbf{A}_1 \mathbf{\Sigma} ],\\
  \psi_{12} &=& (2n)^{-1} \mbox{tr} [ (2 \mathbf{A}_1 ) \mathbf{\Sigma} (\mathbf{W} + \mathbf{W'} ) \mathbf{\Sigma} ]\\
       &=& n^{-1} \mbox{tr} [  \mathbf{A}_1 \mathbf{\Sigma} (\mathbf{W} + \mathbf{W'} ) \mathbf{\Sigma} ],\\
   \psi_{21} &=& \psi_{12},\\
   \psi_{22} &=& (2n)^{-1} \mbox{tr} [ (\mathbf{W} + \mathbf{W'}) \mathbf{\Sigma} 
   (\mathbf{W} + \mathbf{W'})\mathbf{\Sigma} ]
\end{eqnarray*}

For computational purposes, it is important to keep in mind that while the matrices
$\mathbf{A}_{1,2}$ are of dimension $n \times n$, they are typically very sparse.
Furthermore, the matrix $\mathbf{\Sigma}$ is a diagonal matrix, such that post-multiplying a matrix by
$\mathbf{\Sigma}$ amounts to re-scaling the columns of that matrix by the elements on the diagonal
of $\mathbf{\Sigma}$.

\subsubsection{Standard 2SLS Estimation}\label{ss:weightsstandard}
The expression for $\mathbf{a}_r$, with $r = 1,2$, for the case where the estimates
are obtained from 2SLS is given in K-P as follows:
\begin{equation}\label{eq:arstandard}
\mathbf{a}_r = (\mathbf{I} - \lambda \mathbf{W'} )^{-1} \mathbf{HP} \alpha_r,
\end{equation}
with $\mathbf{H}$ as a $n \times p$ matrix of instruments, 
\begin{equation}\label{eq:Pstandard}
\mathbf{P} = (n^{-1}\mathbf{H'H})^{-1} ( n^{-1} \mathbf{H'Z} ) 
        [ (n^{-1} \mathbf{Z'H} ) (n^{-1} \mathbf{H'H} )^{-1} (n^{-1} \mathbf{H'Z}) ]^{-1},
\end{equation}
as a matrix of dimension $p \times k$, and
\begin{equation*}
\alpha_r = - n^{-1} [ \mathbf{Z'} (\mathbf{I} - \lambda \mathbf{W'}) (\mathbf{A}_r + \mathbf{A'}_r) 
 ( \mathbf{I} - \lambda \mathbf{W} ) \mathbf{u} ],
\end{equation*}
where $\alpha_r$ is a vector of dimension $k \times 1$. As a result, $\mathbf{a}_r$ is of
dimension $n \times 1$.

First, let's take a closer look at $\alpha_r$, for $r = 1,2$. Note that $\mathbf{Z'} (\mathbf{I} - \lambda \mathbf{W'})$ can be written in a simpler form as $\mathbf{Z'}_s$. Similarly, 
$ ( \mathbf{I} - \lambda \mathbf{W} ) \mathbf{u}$ is $\mathbf{u}_s$. For both filtered variables,
we use the value of $\lambda$ from the unweighted nonlinear least squares.

For $\alpha_1$, since $\mathbf{A}_1$ is symmetric, $\mathbf{A}_1 + \mathbf{A'}_1 = 2 \mathbf{A}_1$, and the corresponding expression can be written as:
\begin{eqnarray*}
\alpha_1 &=& - 2 n^{-1} [ \mathbf{Z'}_s \mathbf{A}_1 \mathbf{u}_s ]\\
    &=& -2 n^{-1} [ \mathbf{Z'}_{s_L} \mathbf{u}_{s_L} - \mathbf{Z'}_s \mathbf{D} \mathbf{u}_s ],
\end{eqnarray*}
where $\mathbf{Z'}_{s_L}$ and $\mathbf{u}_{s_L}$ are the spatial lags (using weights
matrix $\mathbf{W}$) of respectively the spatially filtered $\mathbf{Z}$ and the spatially
filtered $\mathbf{u}$. For $\alpha_2$, the expression is:
\begin{eqnarray*}
 \alpha_2 &=& - n^{-1} [ \mathbf{Z'}_s (\mathbf{W} + \mathbf{W'} ) \mathbf{u}_s ] \\
    &=& - n^{-1} [ \mathbf{Z'}_s \mathbf{u}_{s_L} + \mathbf{Z'}_{s_L} \mathbf{u}_s ]
\end{eqnarray*}
For easy computation, we will need $\mathbf{Z}_s$ and $\mathbf{u}_s$, as well as their
spatial lags, $\mathbf{Z}_{s_L}$ and $\mathbf{u}_{s_L}$, and the three cross products
$ \mathbf{Z'}_s \mathbf{A}_1 \mathbf{u}_s$, $\mathbf{Z'}_s \mathbf{u}_{s_L} $
and $\mathbf{Z'}_{s_L} \mathbf{u}_s$.

Pre-multiplying the respective expressions for $\alpha_1$ and $\alpha_2$ with the
matrix $\mathbf{P}$ yields a vector of dimension $p \times 1$ ($p$ is the number of
instruments). Pre-multiplying this result with the matrix $\mathbf{H}$ results in a 
vector or dimension $n \times 1$. We will refer to this as vector $\mathbf{v}_r$ (with
$r = 1,2$).


At first sight, the presence of the inverse matrix $( \mathbf{I} - \lambda \mathbf{W'})^{-1}$ in the expression for $\mathbf{a}_r$ in Equation~\ref{eq:arstandard} would seem to preclude large data
analysis. However, we can exploit the approach outlined in \cite{Smirnov:05}. The typical
power expansion (Leontief expansion) of the inverse matrix yields:
\begin{equation*}
 ( \mathbf{I} - \lambda \mathbf{W})^{-1} = \mathbf{I} + \lambda \mathbf{W} + \lambda^2 \mathbf{WW} + \dots
\end{equation*}
As such, this does not help in computation, since the weights matrices involved are still
of dimension $n \times n$. Also, the expression in Equation~\ref{eq:arstandard} pertains to
the transpose of $\mathbf{W}$, i.e., $( \mathbf{I} - \lambda \mathbf{W'} )^{-1}$. In order
to apply the power expansion, it is easier to consider the transpose of $\mathbf{a}_r$:
\begin{equation}
\mathbf{a'}_r = \alpha'_r \mathbf{P'} \mathbf{H'} (\mathbf{I} - \lambda \mathbf{W} )^{-1} = \mathbf{v'}_r (\mathbf{I} - \lambda \mathbf{W} )^{-1},
\end{equation}
where $\mathbf{v'}_r$ is a vector of dimension $1 \times n$.

Using the power expansion yields:
\begin{eqnarray*}
\mathbf{a'}_r &=& \mathbf{v'}_r [  \mathbf{I} + \lambda \mathbf{W} + \lambda^2 \mathbf{WW} + \dots ]\\
  &=& \mathbf{v'}_r + \lambda \mathbf{v'}_r  \mathbf{W}    +\lambda^2  \mathbf{v'}_r  \mathbf{W }\mathbf{W}   + \dots
\end{eqnarray*}
This operation is relatively easy to implement computationally, since once $\lambda \mathbf{v'}_r  \mathbf{W}$ is obtained (a vector of dimension $1 \times n$), all higher order terms consist of
post-multiplying this vector with the (sparse) weights matrix $\mathbf{W}$ and the scalar $\lambda$.
Depending on the
value of $\lambda$, a reasonable approximation is readily obtained for a
relatively low power in the expansion.
For example, a value of $\lambda$ of 0.5 (which is relatively large in practice) reduces to
$0.00098$ after a tenth power.\footnote{For high values of $\lambda$, much higher powers
are needed in order to obtain a reasonable approximation. For example, $0.9$ to the tenth power is still 0.35, but
after fifty powers (i.e., fifty lag operations) it is 0.005.}

With these elements in hand, we obtain the terms $\mathbf{a}_1$ and $\mathbf{a}_2$ needed
in the expression $\mathbf{a'}_q \mathbf{\Sigma} \mathbf{a}_r$, which, together with the trace
terms, yield the four elements of the matrix $\mathbf{\Psi}$.

\subsubsection{Spatially Weighted Estimation}\label{ss:weightsfilter}
When the residuals used in the GMM estimation for $\lambda$ are not the result of a standard
procedure (such as OLS or 2SLS), but instead of a spatially weighted regression (such as
SWLS or GS2SLS), the expressions for the optimal weighting matrix are different in two
respects. The main difference is that the inverse term is no longer present in
Equation~\ref{eq:arstandard} for $\mathbf{a}_r$, which now becomes:
\begin{equation}\label{eq:arweighted}
\mathbf{a}_r = \mathbf{HP} \alpha_r
\end{equation}
The second difference is that the spatially filtered $\mathbf{Z}_s$ are used in the
expression for $\mathbf{P}$ instead of $\mathbf{Z}$. The expression for $\mathbf{P}$
thus becomes:
\begin{equation}\label{eq:filteredP}
\mathbf{P} = (n^{-1}\mathbf{H'H})^{-1} ( n^{-1} \mathbf{H'Z}_s ) 
        [ (n^{-1} \mathbf{Z'}_s \mathbf{H} ) (n^{-1} \mathbf{H'H} )^{-1} (n^{-1} \mathbf{H'Z}_s) ]^{-1}
\end{equation}

\section{Estimation Steps}
For the purposes of this discussion, we will express the model in a generic form as
in Equation~\ref{eq:generic}, which we repeat here:
\begin{equation*}
\mathbf{y} = \mathbf{Z} \delta + \mathbf{u}.
\end{equation*}
This encompasses the two main cases. In the first, no endogenous variables
are present (and thus also no spatially lagged dependent variable) and 
$\mathbf{Z} = \mathbf{X}$ in the usual notation. In the second case, endogenous
variables are present. By convention, we will sort the variables such that the
exogenous variables come first and the endogenous second. In the special
case of a mixed regressive spatial autoregressive model, $\mathbf{Z} = [ \mathbf{X}, \mathbf{Wy} ]$,
and $\delta = [ \beta', \rho]'$.

The actual estimation proceeds in several steps, which are detailed in what follows.

\subsection{Step 1 -- Initial Estimates}
The initial set of estimates, which we will denote as $\delta_1$, are obtained from
either OLS or 2SLS estimation of the model. In case of OLS, this yields:
\begin{equation*}
\delta_{1,OLS} = (\mathbf{Z'Z})^{-1} \mathbf{Z'y},
\end{equation*}
When endogenous variables are present (including the case of a spatially lagged
dependent variable), a matrix of instruments $\mathbf{H}$ is needed and 
estimation follows from:
\begin{equation*}
\delta_{1,2SLS} = (\hat{\mathbf{Z'}} \hat{\mathbf{Z}} )^{-1} \hat{\mathbf{Z'}} \mathbf{y},
\end{equation*}
with $\hat{\mathbf{Z}} = \mathbf{H} (\mathbf{H'H} )^{-1} \mathbf{H'Z}$, or, in one expression, as:
\begin{equation*}
\delta_{1,2SLS} = [ \mathbf{Z'H} (\mathbf{H'H})^{-1} \mathbf{H'Z} ]^{-1} \mathbf{Z'H} (\mathbf{H'H})^{-1} \mathbf{H'} \mathbf{y}.
\end{equation*}

In the presence of a spatially lagged dependent variable, the instruments should 
include multiple orders of spatial lags of the exogenous explanatory variables.
In practice, up to two orders may be sufficient, such that $\mathbf{H} = [ \mathbf{X},
\mathbf{X}_L, \mathbf{X}_{LL} ]$. As always, care must be taken to avoid multicollinear
instruments. For examples, this may be a problem when indicator variables are
included in the model.

The estimates $\delta_1$ yield an initial vector of residuals, $\mathbf{u}_1$ as:
\begin{equation*}
\mathbf{u}_1 = \mathbf{y} - \mathbf{Z} \delta_1.
\end{equation*}

\subsection{Step 2 -- Consistent Estimation of $\lambda$}
A first consistent estimate for $\lambda$, say $\lambda_1$, is obtained
by substituting $\mathbf{u}_1$ into the moment equations of Section~\ref{ss:nonlinearls}
and finding a solution by means of nonlinear least squares.

\subsection{Step 3 -- Efficient and Consistent Estimation of $\lambda$}
An efficient estimate of $\lambda$ is obtained by substituting the values
of $\lambda_1$ and $\mathbf{u}_1$ into the elements of
Equation~\ref{eq:psiqr} as specified in Section~\ref{ss:weightsstandard}.
This yields the weighting matrix $\mathbf{\Psi}(\lambda_1)$, which then allows for a 
weighted nonlinear least squares solution to the moments equations.
This results in the estimate $\lambda_2$.

At this point, we could stop and move to the construction of the asymptotic
variance matrix of the estimates, as outlined in Section~\ref{ss:asyvarstandard}.
For example, this would be relevant if we were only interested in testing
the null hypothesis $\mbox{H}_0: \lambda = 0$.

Typically, however, one does not stop here and moves on to a spatially weighted
estimation of the regression coefficients, which takes into account the consistent
and efficient estimate $\lambda_2$ of the nuisance parameter. Note that only 
consistency of $\lambda$ is required to obtain consistent estimates of the 
$\delta$ coefficients. The use of the more efficient $\lambda_2$ should result in
more efficient estimates of $\delta$ as well, but consistency is not affected by this.

\subsection{Step 4 -- Spatially Weighted Estimation}
The rationale behind spatially weighted estimation is that a simple transformation
(the so-called spatial Cochrane-Orcutt) removes the spatial dependence from
the error term in the regression equation:
\begin{eqnarray*}
(\mathbf{I} - \lambda \mathbf{W})y &=& (\mathbf{I} - \lambda \mathbf{W}) \mathbf{Z} \delta
+ (\mathbf{I} - \lambda \mathbf{W})\mathbf{u}\\
\mathbf{y}_s &=& \mathbf{Z}_s \delta + \varepsilon,
\end{eqnarray*}
where $\mathbf{y}_s$ and $\mathbf{Z}_s$ are filtered variables,
and $\varepsilon$ is a heteroskedastic, but not spatially correlated innovation term.

We distinguish between two situations. In one, there are only exogenous
variables in the regression, so that Spatially Weighted Least Squares (SWLS) is
an appropriate estimation method. In the other, the presence of endogenous
variables requires the use of 2SLS. The special case of a regression with a
spatially lagged dependent variable also falls in this category.

\subsubsection{Spatially Weighted Least Squares -- SWLS}
Spatially Weighted Least Squares is simply OLS applied to the spatially filtered
variables:
\begin{equation}\label{eq:swls}
\delta_{2,SWLS} = ( \mathbf{Z'}_s \mathbf{Z}_s )^{-1} \mathbf{Z'}_s \mathbf{y}_s.
\end{equation}

\subsubsection{Generalized Spatial Two Stage Least Squares - GS2SLS}
Similarly, Generalized Spatial Two Stage Least Squares is 2SLS applied to the
spatially filtered variables:
\begin{equation}\label{eq:gs2sls}
\delta_{2,GS2SLS} = [ \mathbf{Z'}_s \mathbf{H} (\mathbf{H'H})^{-1} \mathbf{H'} \mathbf{Z}_s ]^{-1} \mathbf{Z'}_s \mathbf{H} (\mathbf{H'H})^{-1} \mathbf{H'} \mathbf{y}_s.
\end{equation}
Note that the instrument matrix $\mathbf{H}$ is the same as in Step 1.

\subsubsection{Residuals}
The new estimate coefficient vector $\delta_2$ yields an updated vector of residuals as:
\begin{equation*}
\mathbf{u}_2 = \mathbf{y} - \mathbf{Z} \delta_2.
\end{equation*}
Note that the residuals are computed using the original variables and not the spatially
filtered variables.

\subsection{Step 5 -- Iteration}
The updated residual vector $\mathbf{u}_2$ can now be used to obtain a new
estimate for $\lambda$. Since this is based on a spatially weighted regression,
the proper elements of the weighting matrix $\mathbf{\Psi}$ are given in 
Section~\ref{ss:weightsfilter}, with $\lambda_2$ and $\mathbf{u}_2$ substituted
in the expressions. The solution by means of weighted nonlinear least squares yields
the consistent and efficient estimate $\lambda_3$.

At this point, the value of $\lambda_3$ can be used together with $\delta_2$
to construct an asymptotic variance matrix, as outlined in Section~\ref{ss:asyvarweighted}.
This allows for joint inference on the coefficients $\delta$ and the spatial autoregressive
term $\lambda$.

Alternatively, the new value of $\lambda_3$ could be used in an updated spatially
weighted estimation to yield a new set of estimates for $\delta$ and an associated
residual vector $\mathbf{u}$. These can then be substituted in the moment equations
and in $\mathbf{\Psi}$ to result in a new estimate for $\lambda$. This iteration can be continued
until a suitable stopping criterion is met. To date, there is no evidence as to the 
benefits of iteration beyond $\lambda_3$, but this remains a subject for further investigation.

\section{Asymptotic Variance Matrix}
The asymptotic variance-covariance matrix $\mathbf{\Omega}$ is for both the estimates of $\delta$ and the
estimate of $\lambda$. For the spatial lag model, $\delta$ includes the spatial
autoregressive coefficient $\rho$ (for ease of notation, included as the last element 
in the vector). We distinguish between two major situations. In one, the estimates are
based on Steps 1--3. In other words, $\delta$ are the initial estimates of the regression
model, and $\lambda$ is the consistent and efficient GMM estimate. In the second
case, the estimates are based on a spatially weighted estimation, including 
Steps 4 and 5. For each of these cases, we further distinguish between a model
where only exogenous variables are present (estimated by OLS and SWLS), and
a model where endogenous variables are included (estimated by 2SLS and
GS2SLS). The latter encompasses the spatial lag model as a special case.
In order to carry out inference, $n^{-1} \mathbf{\Omega}$ is used instead of
 $\mathbf{\Omega}$ as such.


To facilitate the expression for the asymptotic variance-covariance matrix,
K-P introduce some additional notation. First, an auxilary $2 \times 1$ vector $\mathbf{J}$
is defined, constructed as:
\begin{equation}\label{eq:J}
\mathbf{J} = \mathbf{G}
\left[
\begin{matrix}
1 \\
2 \lambda
\end{matrix}
\right],
\end{equation}
where the elements of the $2 \times 2$ matrix $\mathbf{G}$ are from Equations
\ref{eq:G11}--\ref{eq:G22}.
K-P further define the overall weighting matrix $\mathbf{\mathbf{\Psi}}_o$ as consisting of four
submatrices, as in:
\begin{equation*}
\mathbf{\mathbf{\Psi}}_o =
\left[
\begin{matrix}
\mathbf{\Psi}_{\delta \delta} & \mathbf{\Psi}_{\delta \lambda}\\
\mathbf{\Psi}'_{\delta \lambda} & \mathbf{\Psi}
\end{matrix}
\right],
\end{equation*}
where $\mathbf{\Psi}$ is the weighting matrix of Equation~\ref{eq:psiqr}, and the other
submatrices differ between the standard estimation (Section~\ref{ss:asyvarstandard})
and the spatially weighted estimation (Section~\ref{ss:asyvarweighted}). Note that the 
dimension of the matrix $\mathbf{\Psi}_o$ is $(p + 2) \times (p + 2)$, where $p$ is the
number of instruments, and $\mathbf{\Psi}$ is of dimension
$2 \times 2$.

The estimator for the variance-covariance matrix takes on the general form:
\begin{equation}\label{eq:genericOmega}
\mathbf{\Omega} =
\left[
\begin{matrix}
\mathbf{P'} & 0\\
0 & (\mathbf{J'} \mathbf{\Psi}^{-1} \mathbf{J} )^{-1} \mathbf{J'} \mathbf{\Psi}^{-1} 
\end{matrix}
\right]
\mathbf{\Psi}_o
\left[
\begin{matrix}
\mathbf{P} & 0\\
0 &  \mathbf{\Psi}^{-1} \mathbf{J}  (\mathbf{J'} \mathbf{\Psi}^{-1} \mathbf{J} )^{-1}  
\end{matrix}
\right],
\end{equation}
where $\mathbf{P}$, of dimension $p \times k$, is constructed differently for each model.
Note that the dimension of the first matrix expression is $(k + 1) \times (p + 2)$, with
the matrix in the lower right hand corner being of dimension $1 \times 2$. The product
of the three matrices yields a result of dimension $(k + 1) \times (k + 1)$.

\subsection{Standard Estimation}\label{ss:asyvarstandard}
The submatrices of $\mathbf{\Psi}_o$ appropriate for the case of standard estimation
are:
\begin{equation}\label{eq:psidd_standard}
\mathbf{\Psi}_{\delta \delta} = n^{-1} \mathbf{H'} (\mathbf{I} - \lambda \mathbf{W} )^{-1} \mathbf{\Sigma}
 (\mathbf{I} - \lambda \mathbf{W'} )^{-1} \mathbf{H}
\end{equation}
and
\begin{equation}\label{eq:psidl_standard}
\mathbf{\Psi}_{\delta \lambda} = n^{-1} \mathbf{H'} (\mathbf{I} - \lambda \mathbf{W} )^{-1} \mathbf{\Sigma} [ \mathbf{a}_1, \mathbf{a}_2 ].
\end{equation}
In these expressions,\footnote{Note that there is a typo in the expression in Equation (B2)
on p. 612 of \cite{Arraizetal:10}. The correct expression for $\mathbf{\Psi}_{\delta \delta}$
and for $\mathbf{\Psi}_{\delta \rho}$ (in their notation) should include the matrix inverse
operation (see A4 on p. 611).}  the residuals from the initial estimation are used, i.e., $\mathbf{u}_1$,
together with the efficient and consistent estimate $\lambda_2$. The diagonal elements
of $\mathbf{\Sigma}$ are the squares of the spatially filtered residuals, $ u_{1is}^2 = (u_{1i} - \lambda_2 u_{1iL} )^{2}$.

\subsubsection{OLS}
In the special case of OLS estimation, $\mathbf{H} = \mathbf{Z}$ (or, $\mathbf{X}$) and the expression
for $\mathbf{P}$ simplifies to $(n^{-1} \mathbf{Z'Z} )^{-1}$. In addition, we have shown that
in this case both $\mathbf{a}_1$ and $\mathbf{a}_2$ are zero. As a result, the weighting
matrix $\mathbf{\Psi}_o$ is block-diagonal

After some matrix algebra, the variance-covariance matrix follows as:
\begin{equation*}
\mathbf{\Omega}_{OLS} =
\left[
\begin{matrix}
n (\mathbf{Z'Z} )^{-1} \mathbf{Z'} ( \mathbf{I} - \lambda \mathbf{W} )^{-1} \mathbf{\Sigma}
 ( \mathbf{I} - \lambda \mathbf{W'} )^{-1} \mathbf{Z} (\mathbf{Z'Z} )^{-1} & 0\\
 0 & ( \mathbf{J'} \mathbf{\Psi}^{-1} \mathbf{J} )^{-1}
 \end{matrix}
\right]
\end{equation*}
Note that the main computational issue in $ \mathbf{Z'} ( \mathbf{I} - \lambda \mathbf{W} )^{-1} $
is post-multiplying the $p \times n$ matrix $\mathbf{Z'}$ with an inverse matrix of dimension
$n \times n$. In practice, we will again exploit the power expansion that was
described in Section~\ref {ss:weightsstandard}.  Once this matrix expression is 
computed, its transpose is  $( \mathbf{I} - \lambda \mathbf{W'} )^{-1} \mathbf{Z}$.


\subsubsection{2SLS}\label{ss:asyvar2sls}
The 2SLS case follows as a straightforward implementation of Equation~\ref{eq:genericOmega},
with $\mathbf{P}$ as in Equation~\ref{eq:Pstandard} (repeated here for convenience):
\begin{equation*}
\mathbf{P} = (n^{-1}\mathbf{H'H})^{-1} ( n^{-1} \mathbf{H'Z} ) 
        [ (n^{-1} \mathbf{Z'H} ) (n^{-1} \mathbf{H'H} )^{-1} (n^{-1} \mathbf{H'Z}) ]^{-1},
\end{equation*}
The submatrices of $\mathbf{\Psi}_o$ are as in Equations~\ref{eq:psidd_standard}
and \ref{eq:psidl_standard}. The residuals are $\mathbf{u}_1$ from the 2SLS regression,
and the spatial parameter is again $\lambda_2$.

\subsection{Spatially Weighted Estimation}\label{ss:asyvarweighted}
In the case where the estimates for $\delta$ are derived from a spatially weighted
regression (yielding $\delta_2$), the relevant submatrices of $\mathbf{\Psi}_o$ take the form:
\begin{equation}\label{eq:psidd_weighted}
\mathbf{\Psi}_{\delta \delta} = n^{-1} \mathbf{H'}  \mathbf{\Sigma}
 \mathbf{H}
\end{equation}
and
\begin{equation}\label{eq:psidl_weighted}
\mathbf{\Psi}_{\delta \lambda} = n^{-1} \mathbf{H'} \mathbf{\Sigma} [ \mathbf{a}_1, \mathbf{a}_2 ].
\end{equation}
The expressions for $\mathbf{a}_1$ and $\mathbf{a}_2$ are as in Equation~\ref{eq:arweighted}, using  $\mathbf{u}_2$ as the residual vector and $\lambda_3$ for the 
autoregressive parameter. The diagonal elements
of $\mathbf{\Sigma}$ are the squares of the spatially filtered residuals, $u_{2is}^2 = (u_{2i} - \lambda_3 u_{2iL} )^{2}$.
Note that it is no longer necessary to compute the inverse matrix $( \mathbf{I} - \lambda \mathbf{W} )^{-1}$. 

\subsubsection{SWLS}
When there are no endogenous variables present in $\mathbf{Z}$, the matrix $\mathbf{\Psi}_o$
is again block-diagonal, since $\mathbf{\Psi}_{\delta \lambda} = 0$. Also, the expression for 
$\mathbf{P}$ simplifies to $(n^{-1} \mathbf{Z'}_s\mathbf{Z}_s )^{-1}$, a cross-product in the
spatially filtered explanatory variables (using $\lambda_3$). Similarly, 
Equation~\ref{eq:psidd_weighted} becomes $\mathbf{\Psi}_{\delta \delta} = n^{-1} \mathbf{Z'}_s \mathbf{\Sigma} \mathbf{Z}_s$.

After some matrix algebra, the variance-covariance matrix follows as:
\begin{equation*}
\mathbf{\Omega}_{SWLS} =
\left[
\begin{matrix}
n (\mathbf{Z}_s'\mathbf{Z}_s )^{-1} \mathbf{Z'}_s \mathbf{\Sigma}
  \mathbf{Z}_s (\mathbf{Z'}_s \mathbf{Z}_s )^{-1} & 0\\
 0 & ( \mathbf{J'} \mathbf{\Psi}^{-1} \mathbf{J} )^{-1}
 \end{matrix}
\right].
\end{equation*}

\subsubsection{GS2SLS}\label{ss:asyvargs2sls}


The relevant expression for $\mathbf{P}$ is as in Equation~\ref{eq:filteredP}, repeated here for convenience:
\begin{equation*}
\mathbf{P} = (n^{-1}\mathbf{H'H})^{-1} ( n^{-1} \mathbf{H'Z}_s ) 
        [ (n^{-1} \mathbf{Z'}_s \mathbf{H} ) (n^{-1} \mathbf{H'H} )^{-1} (n^{-1} \mathbf{H'Z}_s) ]^{-1}
\end{equation*}

The submatrices of $\mathbf{\Psi}_o$ are as in Equations~\ref{eq:psidd_weighted}
and \ref{eq:psidl_weighted}. The residuals are $\mathbf{u}_2$ from the GS2SLS regression,
and the spatial parameter is $\lambda_3$.



\section{Accounting for Heteroskedasticity in FGLS}
The estimation equations for the weighted spatial regression in 
\cite{KelejianPrucha:09} and \cite{Arraizetal:10}  implement the
spatial Cochrane-Orcutt transformation, but do not correct for the
presence of unspecified heteroskedasticity. As shown in \citet[p. 931]{Anselin:06},
among others, the latter could be
implemented by exploiting the
classic result of \cite{White:80} and including a consistent estimate
for $\mathbf{X' \Sigma X}$ or $\mathbf{H' \Sigma H}$ in the second stage
estimation in Step 4. Note that the role of these cross products is explicit
in the estimates for the variance-covariance matrix.

\subsection{SWLS with Heteroskedasticity}
Equation~\ref{eq:swls} is adjusted to account for unspecified heteroskedasticity
and becomes:
\begin{equation*}
\delta_{2,HSWLS} =  [ ( \mathbf{Z'}_s \mathbf{Z}_s ) ( \mathbf{Z'}_s \Sigma \mathbf{Z}_s )^{-1}
 ( \mathbf{Z'}_s \mathbf{Z}_s) ]^{-1} ( \mathbf{Z'}_s \mathbf{Z}_s )  ( \mathbf{Z'}_s \Sigma \mathbf{Z}_s )^{-1} \mathbf{Z'}_s \mathbf{y}_s,
\end{equation*}
with the spatially filtered variables and the spatially filtered residuals based on $\lambda_2$
and $\mathbf{u}_1$.

\subsection{GS2SLS with Heteroskedasticity}
Similarly, Equation~\ref{eq:gs2sls} is adjusted to account for unspecified heteroskedasticity
and becomes:
\begin{equation*}
\delta_{2,HS2SLS} = [ ( \mathbf{Z'}_s \mathbf{H} )(\mathbf{H' \Sigma H})^{-1} ( \mathbf{H'} \mathbf{Z}_s ) ]^{-1} ( \mathbf{Z'}_s \mathbf{H} ) (\mathbf{H' \Sigma H})^{-1} \mathbf{H'} \mathbf{y}_s,
\end{equation*}
again, with the spatially filtered variables and the spatially filtered residuals based on $\lambda_2$
and $\mathbf{u}_1$.

\bibliography{/users/luc/Papers/lucbibs/papers,/users/luc/Papers/lucbibs/luc}
\bibliographystyle{apalike}


\end{document}
